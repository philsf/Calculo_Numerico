\everymath{\displaystyle}
%\documentclass[pdftex,a4paper]{article}
\documentclass[a4paper]{article}
%%classes: article, report, book, proc, amsproc

%%%%%%%%%%%%%%%%%%%%%%%%
%% Misc
% para acertar os acentos
\usepackage[brazilian]{babel} 
%\usepackage[portuguese]{babel} 
% \usepackage[english]{babel}
% \usepackage[T1]{fontenc}
% \usepackage[latin1]{inputenc}
\usepackage[utf8]{inputenc}
\usepackage{indentfirst}
\usepackage{fullpage}
% \usepackage{graphicx} %See PDF section
\usepackage{multicol}
\setlength{\columnseprule}{0.5pt}
\setlength{\columnsep}{20pt}
%%%%%%%%%%%%%%%%%%%%%%%%
%%%%%%%%%%%%%%%%%%%%%%%%
%% PDF support

\usepackage[pdftex]{color,graphicx}
% %% Hyper-refs
\usepackage[pdftex]{hyperref} % for printing
% \usepackage[pdftex,bookmarks,colorlinks]{hyperref} % for screen

%% \newif\ifPDF
%% \ifx\pdfoutput\undefined\PDFfalse
%% \else\ifnum\pdfoutput > 0\PDFtrue
%%      \else\PDFfalse
%%      \fi
%% \fi

%% \ifPDF
%%   \usepackage[T1]{fontenc}
%%   \usepackage{aeguill}
%%   \usepackage[pdftex]{graphicx,color}
%%   \usepackage[pdftex]{hyperref}
%% \else
%%   \usepackage[T1]{fontenc}
%%   \usepackage[dvips]{graphicx}
%%   \usepackage[dvips]{hyperref}
%% \fi

%%%%%%%%%%%%%%%%%%%%%%%%


%%%%%%%%%%%%%%%%%%%%%%%%
%% Math
\usepackage{amsmath,amsfonts,amssymb}
% para usar R de Real do jeito que o povo gosta
\usepackage{amsfonts} % \mathbb
% para usar as letras frescas como L de Espaco das Transf Lineares
% \usepackage{mathrsfs} % \mathscr

% Oferecer seno e tangente em pt, com os comandos usuais.
\providecommand{\sin}{} \renewcommand{\sin}{\hspace{2pt}\mathrm{sen}}
\providecommand{\tan}{} \renewcommand{\tan}{\hspace{2pt}\mathrm{tg}}

% dt of integrals = \ud t
\newcommand{\ud}{\mathrm{\ d}}
%%%%%%%%%%%%%%%%%%%%%%%%



\begin{document}

%%%%%%%%%%%%%%%%%%%%%%%%
%% Título e cabeçalho
%\noindent\parbox[c]{.15\textwidth}{\includegraphics[width=.15\textwidth]{logo}}\hfill
\parbox[c]{.825\textwidth}{\raggedright%
  \sffamily {\LARGE

Cálculo Numérico: Gabarito de Eliminação de Gauss

\par\bigskip}
{Prof: Felipe Figueiredo\par}
{\url{http://sites.google.com/site/proffelipefigueiredo}\par}
}

Versão: \verb|20150519|

%%%%%%%%%%%%%%%%%%%%%%%%


%%%%%%%%%%%%%%%%%%%%%%%%
\section{}


\section{}

\begin{enumerate}
\item % Para cada matriz de coeficientes $A$ e cada vetor $b$ abaixo,
  % resolva o sistema linear $Ax=b$ usando o Método da Eliminação de
  % Gauss:
  \begin{enumerate}
  \item Sistema possível e determinado
    \begin{verbatim}
A1 =
   3.00000   4.00000
   0.00000   0.33333

b1 =
   7.00000
   0.33333

x =
   1.00000
   1.00000
\end{verbatim}
    % $A = \begin{bmatrix}
    %   3 & 4\\
    %   2 & 3\\
    % \end{bmatrix},
    % b= \begin{bmatrix}
    %   7\\
    %   5\\
    % \end{bmatrix}$
    
  \item Sistema possível e determinado
\begin{verbatim}
A1 =
    5.00000    7.00000   11.00000
    0.00000    1.40000    4.20000
    0.00000    1.40000    5.20000

b1 =
   10
   16
    4

A2 =
    5.00000    7.00000   11.00000
    0.00000    1.40000    4.20000
    0.00000    0.00000    1.00000

b2 =
   10
   16
  -12

x =
  -36
   46
  -12
\end{verbatim}
    % $A = \begin{bmatrix}
    %   5 & 7 & 11\\
    %   -1 & 0 & 2\\
    %   -1 & 0 & 3\\
    % \end{bmatrix},
    % b= \begin{bmatrix}
    %   10\\
    %   12\\
    %   0\\
    % \end{bmatrix}$

  \item Sistema impossível (na última etapa: $0x_3 = 3$)
\begin{verbatim}
A1 =
   22.00000  -44.00000    6.00000
    0.00000   17.00000   -2.36364
    0.00000    0.00000    0.00000

b1 =
   2.0000
  -2.9091
   3.0000

A2 =
   22.00000  -44.00000    6.00000
    0.00000   17.00000   -2.36364
    0.00000    0.00000    0.00000

b2 =
   2.0000
  -2.9091
   3.0000
\end{verbatim}
    % $A = \begin{bmatrix}
    %   22 & -44 & 6\\
    %   -5 & 7 & -1\\
    %   -11 & 22 & -3\\
    % \end{bmatrix},
    % b= \begin{bmatrix}
    %   1\\
    %   2\\
    %   -2\\
    % \end{bmatrix}$

  \item Sistema possível e determinado
\begin{verbatim}
A1 =
  -5.00000   0.20000  -0.10000   4.00000
   0.00000   1.24400   1.27800   0.88000
   0.00000  -0.96400   3.68200   0.72000
   0.00000   0.10800  -0.05400  -1.84000

b1 =
   0.00000
   0.10000
  -2.50000
  -5.20000

A2 =
  -5.00000   0.20000  -0.10000   4.00000
   0.00000   1.24400   1.27800   0.88000
   0.00000   0.00000   4.67235   1.40193
   0.00000   0.00000  -0.16495  -1.91640

b2 =
   0.00000
   0.10000
  -2.34502
  -5.21736

A3 =
  -5.00000   0.20000  -0.10000   4.00000
   0.00000   1.24400   1.27800   0.88000
   0.00000   0.00000   4.67235   1.40193
   0.00000   0.00000   0.00000  -1.86691

b3 =
   0.00000
   0.10000
  -2.34502
  -5.30015

x =
   2.27529
  -0.51887
  -1.36936
   2.83582
\end{verbatim}
    % $A = \begin{bmatrix}
    %   -5.0 & 0.2 & -0.1 & 4.0\\
    %   1.1 & 1.2 & 1.3 & 0.0\\
    %   0.9 & -1.0 & 3.7 & 0.0\\
    %   -2.3 & 0.2 & -0.1 & 0.0\\
    % \end{bmatrix},
    % b= \begin{bmatrix}
    %   0.0\\
    %   0.1\\
    %   -2.5\\
    %   -5.2\\
    % \end{bmatrix}$

  \item Sistema possível e determinado
\begin{verbatim}
A1 =
  -3.76000   2.00000   1.71000
   0.00000  -1.50809   0.45479
   0.00000   0.25000   0.37000

b1 =
   0.13000
   0.51915
   0.50000

A2 =
  -3.76000   2.00000   1.71000
   0.00000  -1.50809   0.45479
   0.00000   0.00000   0.44539

b2 =
   0.13000
   0.51915
   0.58606

x =
   0.596089
   0.071612
   1.302965
\end{verbatim}
    % $A = \begin{bmatrix}
    %   -3.76 & 2.00 & 1.71\\
    %   1.00 & -2.04 & 0.00\\
    %   0.00 & 0.25 & 0.37\\
    % \end{bmatrix},
    % b= \begin{bmatrix}
    %   0.13\\
    %   0.45\\
    %   0.50\\
    % \end{bmatrix}$
    
  \item Sistema impossível (na última etapa: $0x_4=4.19902$)
\begin{verbatim}
A1 =
  -1.40000   0.00000   4.00000  -2.00000
   0.00000   2.20000   0.60000  -0.30000
   0.00000   2.00000   5.85714  -2.92857
   0.00000   0.00000  -0.20000   0.10000

b1 =
   2.00000
   0.00000
   5.28571
   4.00000

A2 =
  -1.40000   0.00000   4.00000  -2.00000
   0.00000   2.20000   0.60000  -0.30000
   0.00000   0.00000   5.31169  -2.65584
   0.00000   0.00000  -0.20000   0.10000

b2 =
   2.00000
   0.00000
   5.28571
   4.00000

A3 =
  -1.40000   0.00000   4.00000  -2.00000
   0.00000   2.20000   0.60000  -0.30000
   0.00000   0.00000   5.31169  -2.65584
   0.00000   0.00000   0.00000   0.00000

b3 =
   2.00000
   0.00000
   5.28571
   4.19902
\end{verbatim}
    % $A = \begin{bmatrix}
    %   -1.4 & 0.0 & 4.0 & -2.0\\
    %   -0.7 & 2.2 & 2.6 & -1.3\\
    %   1.0 & 2.0 & 3.0 & -1.5\\
    %   0.0 & 0.0 & -0.2 & 0.1\\
    % \end{bmatrix},
    % b= \begin{bmatrix}
    %   2.0\\
    %   3.0\\
    %   1.0\\
    %   4.0\\
    % \end{bmatrix}$


  % \item Sistema possível e determinado
\begin{verbatim}
\end{verbatim}


  % \item 
  %   $x =
  %   \begin{bmatrix}
  %     \\
  %   \end{bmatrix}$

  \end{enumerate}

\end{enumerate}


\end{document}
