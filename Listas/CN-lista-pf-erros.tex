\everymath{\displaystyle}
%\documentclass[pdftex,a4paper]{article}
\documentclass[a4paper]{article}
%%classes: article, report, book, proc, amsproc

%%%%%%%%%%%%%%%%%%%%%%%%
%% Misc
% para acertar os acentos
\usepackage[brazilian]{babel} 
%\usepackage[portuguese]{babel} 
% \usepackage[english]{babel}
% \usepackage[T1]{fontenc}
% \usepackage[latin1]{inputenc}
\usepackage[utf8]{inputenc}
\usepackage{indentfirst}
\usepackage{fullpage}
% \usepackage{graphicx} %See PDF section
\usepackage{multicol}
\setlength{\columnseprule}{0.5pt}
\setlength{\columnsep}{20pt}
%%%%%%%%%%%%%%%%%%%%%%%%
%%%%%%%%%%%%%%%%%%%%%%%%
%% PDF support

\usepackage[pdftex]{color,graphicx}
% %% Hyper-refs
\usepackage[pdftex]{hyperref} % for printing
% \usepackage[pdftex,bookmarks,colorlinks]{hyperref} % for screen

%% \newif\ifPDF
%% \ifx\pdfoutput\undefined\PDFfalse
%% \else\ifnum\pdfoutput > 0\PDFtrue
%%      \else\PDFfalse
%%      \fi
%% \fi

%% \ifPDF
%%   \usepackage[T1]{fontenc}
%%   \usepackage{aeguill}
%%   \usepackage[pdftex]{graphicx,color}
%%   \usepackage[pdftex]{hyperref}
%% \else
%%   \usepackage[T1]{fontenc}
%%   \usepackage[dvips]{graphicx}
%%   \usepackage[dvips]{hyperref}
%% \fi

%%%%%%%%%%%%%%%%%%%%%%%%


%%%%%%%%%%%%%%%%%%%%%%%%
%% Math
\usepackage{amsmath,amsfonts,amssymb}
% para usar R de Real do jeito que o povo gosta
\usepackage{amsfonts} % \mathbb
% para usar as letras frescas como L de Espaco das Transf Lineares
% \usepackage{mathrsfs} % \mathscr

% Oferecer seno e tangente em pt, com os comandos usuais.
\providecommand{\sin}{} \renewcommand{\sin}{\hspace{2pt}\mathrm{sen}}
\providecommand{\tan}{} \renewcommand{\tan}{\hspace{2pt}\mathrm{tg}}

% dt of integrals = \ud t
\newcommand{\ud}{\mathrm{\ d}}
%%%%%%%%%%%%%%%%%%%%%%%%



\begin{document}

%%%%%%%%%%%%%%%%%%%%%%%%
%% Título e cabeçalho
%\noindent\parbox[c]{.15\textwidth}{\includegraphics[width=.15\textwidth]{logo}}\hfill
\parbox[c]{.825\textwidth}{\raggedright%
  \sffamily {\LARGE

Cálculo Numérico: Lista de Ponto Flutuante e Erros

\par\bigskip}
{Prof: Felipe Figueiredo\par}
{\url{http://sites.google.com/site/proffelipefigueiredo}}

\vspace{1cm}
}
%%%%%%%%%%%%%%%%%%%%%%%%


%%%%%%%%%%%%%%%%%%%%%%%%
\section{Formulário}

Forma normalizada do Ponto Flutuante numa máquina com a configuração $f(\beta,t,m,M)$

\begin{displaymath}
  \pm 0.ddd \ldots dd \times \beta ^e  \mathrm {,\ onde } -m \le e \le M
\end{displaymath}

%onde $-m \le e \le M$

Erro Absoluto:
\begin{displaymath}
  EA = x - \bar{x}
\end{displaymath}

Erro Relativo:

\begin{displaymath}
  ER = \frac{x-\bar{x}}{\bar{x}}
\end{displaymath}
\section{Exercícios}

\begin{enumerate}
\item Normalize a representação em ponto flutuante dos seguintes
  números:
  \begin{enumerate}
  \item $30$ % R: $0.3 \times 10^2$
  \item $520000$ % R: $0.52 \times 10^6$
  \item $\frac{1}{1000}$ % R:  $0.001 = 0.1 \times 10^{-2}$
  \item $1250.075$ % R: $0.1250075 \times 10^{4}$
  \item $10^{-2}$ % R: $0.01 = 0.1 \times 10^{-1}$
  \item $0.000347779$ % R:  $0.347779  \times  10^{-3}$
  \item $3.925 \times 10^{-6}$ % R:  $0.3925 \times 10^{-5}$
  \end{enumerate}

\item Encontre o menor e o maior número possível em cada máquina a
  seguir:
  \begin{enumerate}
  \item $f(10,2,3,4)$ % R: menor: $0.10 \times 10^{-3}$, maior: $0.99 \times 10^4$
  \item $f(10,2,4,3)$ % R: menor: $0.10 \times 10^{-4}$, maior: $0.99 \times 10^3$
  \item $f(10,6,2,2)$ % R: menor: $0.100000 \times 10^{-2}$, maior: $0.999999 \times 10^2$
  \item $f(2,8,3,3)$ % R: menor: $0.10000000 \times 10^{-3}$, maior: $0.11111111 \times 10^3$
  % \item $f(10,7,?,?)$
  % \item $f(10,16,?,?)$
  \end{enumerate}

% 100000001490116119384765625

\item Encontre a representação aproximada tanto por truncamento como
  por arredondamento dos números do exercício 1 na máquina
  $f(10,3,4,5)$:
  % \begin{enumerate}
  % \item $30$ %  R: truncamento: $\bar{x}=0.300 \times 10^2$ arredondamento: $\bar{x}=0.300 \times 10^2$
  % \item $520000$  R: overflow
  % \item $\frac{1}{1000}$ % R: truncamento: $\bar{x}=0.100 \times 10^{-2}$ arredondamento: $\bar{x}=0.100 \times 10^{-2}$
  % \item $1250.075$ % R: truncamento: $\bar{x} = 0.125 \times 10^{4}$ arredondamento: $\bar{x} = 0.125 \times 10^{4}$
  % \item $10^{-2}$ % R: truncamento: $\bar{x} = 0.100 \times 10^{-1}$ arredondamento: $\bar{x} = 0.100 \times 10^{-1}$
  % \item $0.000347779$ % R: $ truncamento: \bar{x} = 0.347  \times  10^{-3}$ arredondamento: \bar{x} = 0.348  \times  10^{-3}$
  % \item $3.925 \times 10^{-6}$ % R: underflow
  % \end{enumerate}

\item Determine o erro absoluto para as aproximações de truncamento do
  exercício anterior.
  % \begin{enumerate}
  % \item $30$ %  R: truncamento: $\bar{x}=0.300 \times 10^2$ 
  % \item $520000$  R: overflow
  % \item $\frac{1}{1000}$ % R: truncamento: $\bar{x}=0.100 \times 10^{-2}$ 
  % \item $1250.075$ % R: truncamento: $\bar{x} = 0.125 \times 10^{4}$ 
  % \item $10^{-2}$ % R: truncamento: $\bar{x} = 0.100 \times 10^{-1}$ 
  % \item $0.000347779$ % R: $ truncamento: \bar{x} = 0.347  \times  10^{-3}$ 
  % \item $3.925 \times 10^{-6}$ % R: underflow
  % \end{enumerate}



  \section{Problemas}

  
% \item O padrão IEEE754 determina que com precisão simples (32-bits)
%     a mantissa tem 24 bits e o expoente 8 bits e com precisão dupla
%     (64-bits) a mantissa tem 53 bits e o expoente tem 11 bits.
%     \begin{enumerate}
%     \item (precisão simples: 32-bits) $f(2,24,126,127)$ % menor: , maior: 
%     \item (precisão dupla: 64-bits) $f(2,53,1022,1023)$ % menor: , maior: 
%     \end{enumerate}

% \item (ULP: Espaço entre dois números) $\varepsilon$ da máquina.

\item Considerando uma máquina com configuração $f(10,2,3,4)$,
  encontre o erro relativo da representação do número $317$:
  \begin{enumerate}
  \item Utilizando truncamento % $0.31 \times 10^3$ $ER = \frac{0.317
                               % - 0.31}{0.31} = 0.022580645$
  \item Utilizando arredondamento % $0.32 \times 10^3$ $ER =
                                % \frac{0.317 - 0.32}{0.32} = −0,009375$
  \end{enumerate}
  
\item (Desafio) Quantos números podem ser representados na forma
  normalizada numa máquina com configuração $f(2,1,1,1)$? E se a
  máquina tiver a configuração $f(2,2,1,1)$?

\end{enumerate}
\end{document}
