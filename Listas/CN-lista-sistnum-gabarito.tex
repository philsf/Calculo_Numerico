\everymath{\displaystyle}
%\documentclass[pdftex,a4paper]{article}
\documentclass[a4paper]{article}
%%classes: article, report, book, proc, amsproc

%%%%%%%%%%%%%%%%%%%%%%%%
%% Misc
% para acertar os acentos
\usepackage[brazilian]{babel} 
%\usepackage[portuguese]{babel} 
% \usepackage[english]{babel}
% \usepackage[T1]{fontenc}
% \usepackage[latin1]{inputenc}
\usepackage[utf8]{inputenc}
\usepackage{indentfirst}
\usepackage{fullpage}
% \usepackage{graphicx} %See PDF section
\usepackage{multicol}
\setlength{\columnseprule}{0.5pt}
\setlength{\columnsep}{20pt}
%%%%%%%%%%%%%%%%%%%%%%%%
%%%%%%%%%%%%%%%%%%%%%%%%
%% PDF support

\usepackage[pdftex]{color,graphicx}
% %% Hyper-refs
\usepackage[pdftex]{hyperref} % for printing
% \usepackage[pdftex,bookmarks,colorlinks]{hyperref} % for screen

%% \newif\ifPDF
%% \ifx\pdfoutput\undefined\PDFfalse
%% \else\ifnum\pdfoutput > 0\PDFtrue
%%      \else\PDFfalse
%%      \fi
%% \fi

%% \ifPDF
%%   \usepackage[T1]{fontenc}
%%   \usepackage{aeguill}
%%   \usepackage[pdftex]{graphicx,color}
%%   \usepackage[pdftex]{hyperref}
%% \else
%%   \usepackage[T1]{fontenc}
%%   \usepackage[dvips]{graphicx}
%%   \usepackage[dvips]{hyperref}
%% \fi

%%%%%%%%%%%%%%%%%%%%%%%%


%%%%%%%%%%%%%%%%%%%%%%%%
%% Math
\usepackage{amsmath,amsfonts,amssymb}
% para usar R de Real do jeito que o povo gosta
\usepackage{amsfonts} % \mathbb
% para usar as letras frescas como L de Espaco das Transf Lineares
% \usepackage{mathrsfs} % \mathscr

% Oferecer seno e tangente em pt, com os comandos usuais.
\providecommand{\sin}{} \renewcommand{\sin}{\hspace{2pt}\mathrm{sen}}
\providecommand{\tan}{} \renewcommand{\tan}{\hspace{2pt}\mathrm{tg}}

% dt of integrals = \ud t
\newcommand{\ud}{\mathrm{\ d}}
%%%%%%%%%%%%%%%%%%%%%%%%



\begin{document}

%%%%%%%%%%%%%%%%%%%%%%%%
%% Título e cabeçalho
%\noindent\parbox[c]{.15\textwidth}{\includegraphics[width=.15\textwidth]{logo}}\hfill
\parbox[c]{.825\textwidth}{\raggedright%
  \sffamily {\LARGE

Cálculo Numérico: Gabarito de Sistemas de Numeração

\par\bigskip}
{Prof: Felipe Figueiredo\par}
{\url{http://sites.google.com/site/proffelipefigueiredo}\par}
}

Versão: \verb|20150323|

%%%%%%%%%%%%%%%%%%%%%%%%


%%%%%%%%%%%%%%%%%%%%%%%%
% \section{Formulário}

\section{Exercícios}

\begin{enumerate}
\item Converta a representação dos seguintes números de decimal para binário:

  \begin{enumerate}
  \item $1010_{2}$
  \item $1011_{2}$
  \item $1000110_{2}$
  \item $1000011_{2}$
  \item $0.001100110011\ldots_{2}$
  \item $0.1101_{2}$
  \item $0.1001_{2}$
  \item $0.011100110011001\ldots_{2}$
  \item $1.1_{2}$
  \item $1011.101_{2}$
  \item $1000110.1101_{2}$
  \end{enumerate}

\item Converta a representação dos seguintes números de binário para
  decimal:
  \begin{enumerate}
  \item $9_{10}$
  \item $21_{10}$
  \item $26_{10}$
  \item $33_{10}$
  \item $127_{10}$
  \item $255_{10}$
  \end{enumerate}

\item Converta a representação dos seguintes números de hexadecimal
  para decimal:
  \begin{enumerate}
  \item $74_{10}$
  \item $90_{10}$
  \item $255_{10}$
  \item $254_{10}$
  \item $2016_{10}$
  \end{enumerate}

\item Converta a representação dos seguintes números de decimal para
  hexadecimal:
  \begin{enumerate}
  \item $3C_{16}$
  \item $B2_{16}$
  \item $ABBA_{16}$
  \item $21_{16}$
  \item $90_{16}$
  \end{enumerate}
\end{enumerate}

\end{document}
