\everymath{\displaystyle}
%\documentclass[pdftex,a4paper]{article}
\documentclass[a4paper]{article}
%%classes: article, report, book, proc, amsproc

%%%%%%%%%%%%%%%%%%%%%%%%
%% Misc
% para acertar os acentos
\usepackage[brazilian]{babel} 
%\usepackage[portuguese]{babel} 
% \usepackage[english]{babel}
% \usepackage[T1]{fontenc}
% \usepackage[latin1]{inputenc}
\usepackage[utf8]{inputenc}
\usepackage{indentfirst}
\usepackage{fullpage}
% \usepackage{graphicx} %See PDF section
\usepackage{multicol}
\setlength{\columnseprule}{0.5pt}
\setlength{\columnsep}{20pt}
%%%%%%%%%%%%%%%%%%%%%%%%
%%%%%%%%%%%%%%%%%%%%%%%%
%% PDF support

\usepackage[pdftex]{color,graphicx}
% %% Hyper-refs
\usepackage[pdftex]{hyperref} % for printing
% \usepackage[pdftex,bookmarks,colorlinks]{hyperref} % for screen

%% \newif\ifPDF
%% \ifx\pdfoutput\undefined\PDFfalse
%% \else\ifnum\pdfoutput > 0\PDFtrue
%%      \else\PDFfalse
%%      \fi
%% \fi

%% \ifPDF
%%   \usepackage[T1]{fontenc}
%%   \usepackage{aeguill}
%%   \usepackage[pdftex]{graphicx,color}
%%   \usepackage[pdftex]{hyperref}
%% \else
%%   \usepackage[T1]{fontenc}
%%   \usepackage[dvips]{graphicx}
%%   \usepackage[dvips]{hyperref}
%% \fi

%%%%%%%%%%%%%%%%%%%%%%%%


%%%%%%%%%%%%%%%%%%%%%%%%
%% Math
\usepackage{amsmath,amsfonts,amssymb}
% para usar R de Real do jeito que o povo gosta
\usepackage{amsfonts} % \mathbb
% para usar as letras frescas como L de Espaco das Transf Lineares
% \usepackage{mathrsfs} % \mathscr

% Oferecer seno e tangente em pt, com os comandos usuais.
\providecommand{\sin}{} \renewcommand{\sin}{\hspace{2pt}\mathrm{sen}}
\providecommand{\tan}{} \renewcommand{\tan}{\hspace{2pt}\mathrm{tg}}

% dt of integrals = \ud t
\newcommand{\ud}{\mathrm{\ d}}
%%%%%%%%%%%%%%%%%%%%%%%%



\begin{document}

%%%%%%%%%%%%%%%%%%%%%%%%
%% Título e cabeçalho
%\noindent\parbox[c]{.15\textwidth}{\includegraphics[width=.15\textwidth]{logo}}\hfill
\parbox[c]{.825\textwidth}{\raggedright%
  \sffamily {\LARGE

Cálculo Numérico: Lista Método da Bissecção

\par\bigskip}
{Prof: Felipe Figueiredo\par}
{\url{http://sites.google.com/site/proffelipefigueiredo}}

\vspace{1cm}
}
%%%%%%%%%%%%%%%%%%%%%%%%


%%%%%%%%%%%%%%%%%%%%%%%%
\section{Formulário}

\subsection*{Teste}

\begin{center}
  \fbox{
  % \addtolength{\linewidth}{-2\fboxsep}%
  % \addtolength{\linewidth}{-2\fboxrule}%
    \begin{minipage}{0.3\linewidth}
      Se $f(a)f(x)>0$, então $a=x$.  

      Caso contrário, então $b=x$.
    \end{minipage}
  }
\end{center}

\subsection*{Critérios de parada}

\begin{enumerate}
\item Número máximo de iterações (passos) $k$

\item Precisão: erro máximo $\varepsilon$

\begin{displaymath}
  \varepsilon = b - a
\end{displaymath}

\end{enumerate}

\section{Exercícios}

\begin{enumerate}
\item Encontre uma aproximação para a raiz contida em cada intervalo
  abaixo para cada função. Use o método da bissecção até atingir a
  precisão de $\varepsilon<10^{-2}$ ou $k=4$ passos, o que ocorrer
  primeiro. Identifique na sua resposta a sequência $x_k$ obtida, e
  use o último $x_k$ como resposta aproximada $\bar{x}$:

  \begin{enumerate}
  \item $f(x) = x$, em $[-1, 3]$
  \item $f(x) = x$, em $[-1, 4]$
  \item $f(x) = x^2 - 4$, em $[1.5, 3]$
  \item $f(x) = x^3$, em $[-0.5, 1]$
  \item $f(x) = x^3-1.5x$, em $[1, 2.5]$
  \item $f(x) = x e^x$, em $[-0.5, 1]$
  \item $f(x) = \sin x$, em
    $\left[\frac{-\pi}{2},\frac{\pi}{4}\right]$
  % \item $f(x) = $, em $[]$
  % \item $f(x) = $, em $[]$
  \end{enumerate}

\item Determine o erro absoluto e o erro relativo da aproximação
  $\bar{x}$ encontrada em cada item do exercício 1, considerando que
  as soluções exatas são:
  \begin{enumerate}
  \item $x=0$ %$f(x) = x$, em $[-1, 3]$
  \item $x=0$ %$f(x) = x$, em $[-1, 4]$
  \item $x=2$ %$f(x) = x^2 - 4$, em $[1.5, 3]$
  \item $x=0$ %$f(x) = x^3$, em $[-0.5, 1]$
  \item $x=\sqrt{1.5}$ %$f(x) = x^3-1.5x$, em $[1, 2.5]$
  \item $x=0$ %$f(x) = x e^x$, em $[-0.5, 1]$
  \item $x=0$ %$f(x) = \sin x$, em $\left[\frac{-\pi}{2},\frac{\pi}{2}\right]$
  % \item $x=$ %$f(x) = $, em $[]$
  % \item $x=$ %$f(x) = $, em $[]$
  \end{enumerate}

% \item Determine o erro relativo da aproximação $\bar{x}$ encontrada em
%   cada item do exercício 1, considerando que as soluções exatas são as
%   do exercício 2:

\item Faça o estudo de sinais das funções do exercício 1 e isole as
  raízes em intervalos que contenham uma única raiz, cada. (Obs:
  existem várias respostas possíveis. Encontre a sua!)

\section{Problemas}

% Encontre o erro relativo da solução encontrada usando o método da
% bissecção após 3 iterações no intervalo [a,b]

\item O problema ``encontrar a raiz de $f(x) = x^2 e^x$ em
  $[-0.3,1.2]$'' não pode ser resolvido pelo método da bissecção. Por
  que?

\item (Desafio) O método da bissecção pode ser utilizado para
  encontrar uma aproximação de $\sqrt{2}$. Pense e identifique uma
  função apropriada que tenha como raiz o número $\sqrt{2}$ e use um
  intervalo apropriado para aplicar o método.

\item (Desafio) Quantas iterações do método da bissecção são
  necessárias para atingir a precisão $\varepsilon<10^{-3}$ para a
  função $f(x)= \frac{\cos x}{| x \log x |}$ no intervalo $\left[\pi,
    2\pi \right]$?

\item (Desafio: baseado em fatos reais) Você trabalha como
  administrador Linux de um servidor de emails, mas após uma pane
  elétrica o arquivo de emails do gerente foi corrompido. Esse cliente
  precisa com urgência das informações contidas em um email específico
  antes de uma reunião, mas depende de você para recuperá-lo. Ele lhe
  fornece uma palavra chave específica que ele garante estar contida
  apenas no email que ele precisa, e pode ser usada para buscá-lo na
  inbox. Como o a inbox do usuário é um arquivo texto com 10732891
  linhas, você não pode simplesmente abrí-lo num editor de textos e
  fazer a busca pois a memória do computador não suporta um arquivo
  deste tamanho. Você decide usar um comando do UNIX que lhe permite
  localizar uma palavra em arquivos de qualquer tamanho de maneira
  eficiente (quase instantânea) e decide usar isso para cortar o
  arquivo pela metade várias vezes até que o arquivo final seja
  pequeno o suficiente para ser aberto e o email recuperado. Quantas
  iterações desse processo serão necessárias para que o arquivo final
  tenha menos de 10000 linhas?
\end{enumerate}


\end{document}
