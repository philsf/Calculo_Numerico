\everymath{\displaystyle}
%\documentclass[pdftex,a4paper]{article}
\documentclass[a4paper]{article}
%%classes: article, report, book, proc, amsproc

%%%%%%%%%%%%%%%%%%%%%%%%
%% Misc
% para acertar os acentos
\usepackage[brazilian]{babel} 
%\usepackage[portuguese]{babel} 
% \usepackage[english]{babel}
% \usepackage[T1]{fontenc}
% \usepackage[latin1]{inputenc}
\usepackage[utf8]{inputenc}
\usepackage{indentfirst}
\usepackage{fullpage}
% \usepackage{graphicx} %See PDF section
\usepackage{multicol}
\setlength{\columnseprule}{0.5pt}
\setlength{\columnsep}{20pt}
%%%%%%%%%%%%%%%%%%%%%%%%
%%%%%%%%%%%%%%%%%%%%%%%%
%% PDF support

\usepackage[pdftex]{color,graphicx}
% %% Hyper-refs
\usepackage[pdftex]{hyperref} % for printing
% \usepackage[pdftex,bookmarks,colorlinks]{hyperref} % for screen

%% \newif\ifPDF
%% \ifx\pdfoutput\undefined\PDFfalse
%% \else\ifnum\pdfoutput > 0\PDFtrue
%%      \else\PDFfalse
%%      \fi
%% \fi

%% \ifPDF
%%   \usepackage[T1]{fontenc}
%%   \usepackage{aeguill}
%%   \usepackage[pdftex]{graphicx,color}
%%   \usepackage[pdftex]{hyperref}
%% \else
%%   \usepackage[T1]{fontenc}
%%   \usepackage[dvips]{graphicx}
%%   \usepackage[dvips]{hyperref}
%% \fi

%%%%%%%%%%%%%%%%%%%%%%%%


%%%%%%%%%%%%%%%%%%%%%%%%
%% Math
\usepackage{amsmath,amsfonts,amssymb}
% para usar R de Real do jeito que o povo gosta
\usepackage{amsfonts} % \mathbb
% para usar as letras frescas como L de Espaco das Transf Lineares
% \usepackage{mathrsfs} % \mathscr

% Oferecer seno e tangente em pt, com os comandos usuais.
\providecommand{\sin}{} \renewcommand{\sin}{\hspace{2pt}\mathrm{sen}}
\providecommand{\tan}{} \renewcommand{\tan}{\hspace{2pt}\mathrm{tg}}

% dt of integrals = \ud t
\newcommand{\ud}{\mathrm{\ d}}
%%%%%%%%%%%%%%%%%%%%%%%%

\begin{document}

%%%%%%%%%%%%%%%%%%%%%%%%
%% Título e cabeçalho
%\noindent\parbox[c]{.15\textwidth}{\includegraphics[width=.15\textwidth]{logo}}\hfill
\parbox[c]{.825\textwidth}{\raggedright%
  \sffamily {\LARGE

Cálculo Numérico: Gabarito de Integração Numérica

\par\bigskip}
{Prof: Felipe Figueiredo\par}
{\url{http://sites.google.com/site/proffelipefigueiredo}\par}
}

Versão: \verb|20150526|

%%%%%%%%%%%%%%%%%%%%%%%%


%%%%%%%%%%%%%%%%%%%%%%%%
\section{}

\section{Exercícios}

\begin{enumerate}
\item
  \begin{enumerate}
  \item 2.3438 %$f(x)= x^2$, em $[1,2]$, com 4 subdivisões 
    % x=1:.25:2, f=x.^2, trapezios(f,[1 2],4)
  \item 1.2185 %$f(x) = \sqrt{x}$ em $[1,2]$, com 5 subdivisões
    %x=1:.2:2, f=sqrt(x), trapezios(f,[1 2],5)
  \item 2.4849 %$f(x) = \ln(x)$, em $[1,4]$, com 3 subdivisões
    % x=1:4, f=log(x), trapezios(f,[1 4],3)
  \item 0 %$f(x) = \sin x$, em $[-3.14,3.14]$, com 4 subdivisões
    % x=-3.14:1.57:3.14, f=sin(x), trapezios(f,[-3.14 3.14],4)
  \item 4.3713 %$f(x) = 2^x$, em $[0,2]$, com 4 subdivisões
    % x=0:0.5:2, f=2.^x, trapezios(f,[0 2],4)
  \item 0.70833 %$f(x) = \frac{1}{1+x}$, em $[0,1]$, com 2 subdivisões
    % x=0:0.5:1, f=1./(1+x), trapezios(f,[0 1],2)
  \end{enumerate}

\item 
  \begin{enumerate}
  \item 0.22807 %$\int_1^{1.5}x^2\ln x \ud x$
    % x=[1 1.5], f=x.^2 .*log(x), trapezios(f,[1 1.5],1)
  \item -0.86667 %$\int_{1}^{1.6} \frac{2x}{x^2-4}\ud x$
    % x=[1 1.6], f=2.*x./(x.^2-4), trapezios(f,[1 1.6],1)
  \item 0.46940 %$\int_{-0.25}^{0.25}\cos ^2 x \ud x$
    % x=[-0.25 0.25], f=cos(x).^2, trapezios(f,[-0.25 0.25],1)
  \end{enumerate}

\section{Problemas}
\item 0.23015
% x=0:0.5:2, f=sin(x).^2.*cos(x).^2, trapezios(f,[0 2],4)

\item 1.4907
% x=0:0.25:1, f=exp(x.^2), trapezios(f,[0 1],4)

\item 
  \begin{enumerate}
  \item 0.75000
    % x=[1 2], f= 1./x, trapezios(f,[1 2],1)

  \item 0.70833
    % x=1:0.5:2, f= 1./x, trapezios(f,[1 2],2)

  \item $\ln 2 = 0.693147181$, EA do item (a): -0.056853, EA do item
    (b): -0.015186
  \end{enumerate}

\item Utilizando 5 trapézios, encontra-se o comprimento 33.213cm
% x=0:10:50, f=sqrt(1-cos(x).^2), trapezios(f,[0 50],5)

\item 2958m
% x=0:6:84
% f=[37 40 45 47 44 40 36 33 30 25 23 27 31 35 37 ]
% trapezios(f,[0 84],14)
% original (ft/s): f=[124 134 148 156 147 133 121 109 99 85 78 89 104 116 123]

\end{enumerate}


\end{document}
