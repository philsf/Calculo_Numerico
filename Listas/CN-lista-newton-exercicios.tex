\everymath{\displaystyle}
%\documentclass[pdftex,a4paper]{article}
\documentclass[a4paper]{article}
%%classes: article, report, book, proc, amsproc

%%%%%%%%%%%%%%%%%%%%%%%%
%% Misc
% para acertar os acentos
\usepackage[brazilian]{babel} 
%\usepackage[portuguese]{babel} 
% \usepackage[english]{babel}
% \usepackage[T1]{fontenc}
% \usepackage[latin1]{inputenc}
\usepackage[utf8]{inputenc}
\usepackage{indentfirst}
\usepackage{fullpage}
% \usepackage{graphicx} %See PDF section
\usepackage{multicol}
\setlength{\columnseprule}{0.5pt}
\setlength{\columnsep}{20pt}
%%%%%%%%%%%%%%%%%%%%%%%%
%%%%%%%%%%%%%%%%%%%%%%%%
%% PDF support

\usepackage[pdftex]{color,graphicx}
% %% Hyper-refs
\usepackage[pdftex]{hyperref} % for printing
% \usepackage[pdftex,bookmarks,colorlinks]{hyperref} % for screen

%% \newif\ifPDF
%% \ifx\pdfoutput\undefined\PDFfalse
%% \else\ifnum\pdfoutput > 0\PDFtrue
%%      \else\PDFfalse
%%      \fi
%% \fi

%% \ifPDF
%%   \usepackage[T1]{fontenc}
%%   \usepackage{aeguill}
%%   \usepackage[pdftex]{graphicx,color}
%%   \usepackage[pdftex]{hyperref}
%% \else
%%   \usepackage[T1]{fontenc}
%%   \usepackage[dvips]{graphicx}
%%   \usepackage[dvips]{hyperref}
%% \fi

%%%%%%%%%%%%%%%%%%%%%%%%


%%%%%%%%%%%%%%%%%%%%%%%%
%% Math
\usepackage{amsmath,amsfonts,amssymb}
% para usar R de Real do jeito que o povo gosta
\usepackage{amsfonts} % \mathbb
% para usar as letras frescas como L de Espaco das Transf Lineares
% \usepackage{mathrsfs} % \mathscr

% Oferecer seno e tangente em pt, com os comandos usuais.
\providecommand{\sin}{} \renewcommand{\sin}{\hspace{2pt}\mathrm{sen}}
\providecommand{\tan}{} \renewcommand{\tan}{\hspace{2pt}\mathrm{tg}}

% dt of integrals = \ud t
\newcommand{\ud}{\mathrm{\ d}}
%%%%%%%%%%%%%%%%%%%%%%%%



\begin{document}

%%%%%%%%%%%%%%%%%%%%%%%%
%% Título e cabeçalho
%\noindent\parbox[c]{.15\textwidth}{\includegraphics[width=.15\textwidth]{logo}}\hfill
\parbox[c]{.825\textwidth}{\raggedright%
  \sffamily {\LARGE

Cálculo Numérico: Lista de Método de Newton

\par\bigskip}
{Prof: Felipe Figueiredo\par}
{\url{http://sites.google.com/site/proffelipefigueiredo}}

\vspace{1cm}
}
%%%%%%%%%%%%%%%%%%%%%%%%


%%%%%%%%%%%%%%%%%%%%%%%%
\section{Formulário}

\subsection*{Sequência}

\begin{displaymath}
  x_{k+1} = x_k - \frac{f(x_k)}{f'(x_k)}
\end{displaymath}

\subsection*{Critérios de parada}

\begin{enumerate}
\item Número máximo de iterações (passos) $k$

\item Precisão: distância entre duas aproximações consecutivas
  $\varepsilon$ 
  \begin{displaymath}
    \varepsilon = |x_k - x_{k-1}|
  \end{displaymath}

\item Precisão: valor absoluto da função $\varepsilon$
  \begin{displaymath}
    \varepsilon = |f(x_k)|
  \end{displaymath}

\end{enumerate}

\section{Exercícios}

\begin{enumerate}
\item Encontre uma aproximação para a raiz das funções abaixo, com
  cada ponto inicial dado. Use o método de Newton até atingir a
  precisão de $\varepsilon<10^{-2}$ ou $k=4$ passos, o que ocorrer
  primeiro. Identifique na sua resposta a sequência $x_k$ obtida, e
  use o último $x_k$ como resposta aproximada $\bar{x}$:

  \begin{enumerate}
  % \item $f(x) = $ com $x_0 = $
  % \item $f(x) = $ com $x_0 = $
  % \item $f(x) = $ com $x_0 = $
  % \item $f(x) = $ com $x_0 = $
  % \item $f(x) = $ com $x_0 = $
  % \item $f(x) = $ com $x_0 = $
  % \item $f(x) = $ com $x_0 = $
  % \item $f(x) = $ com $x_0 = $
  % \item $f(x) = $ com $x_0 = $
  \item $f(x) = x$, com $x_0 = 1.5$
  \item $f(x) = x$, com $x_0 = 10$
  \item $f(x) = x^2 - 4$, com $x_0 = 5$
  \item $f(x) = x^3$, com $x_0 = -3$
  \item $f(x) = x^3-1.5x$, com $x_0 = 6$
  \item $f(x) = x e^x$, com $x_0 = 1.1$
  \item $f(x) = \sin x$, com $x_0 = 1$
    %$\left[\frac{-\pi}{2},\frac{\pi}{4}\right]$
  \end{enumerate}

\item Determine o erro absoluto e o erro relativo da aproximação
  $\bar{x}$ encontrada em cada item do exercício 1, considerando que
  as soluções exatas são:
  \begin{enumerate}
  \item $x=0$ %$f(x) = x$, em $[-1, 3]$
  \item $x=0$ %$f(x) = x$, em $[-1, 4]$
  \item $x=2$ %$f(x) = x^2 - 4$, em $[1.5, 3]$
  \item $x=0$ %$f(x) = x^3$, em $[-0.5, 1]$
  \item $x=\sqrt{1.5}$ %$f(x) = x^3-1.5x$, em $[1, 2.5]$
  \item $x=0$ %$f(x) = x e^x$, em $[-0.5, 1]$
  \item $x=0$ %$f(x) = \sin x$, em $\left[\frac{-\pi}{2},\frac{\pi}{2}\right]$
  % \item $x=$ %$f(x) = $, em $[]$
  % \item $x=$ %$f(x) = $, em $[]$
  \end{enumerate}

\section{Problemas}

\item (Comparação entre Bissecção e Newton) Entenda como se compara a
  eficiência entre os métodos da Bissecção e Newton.

  \begin{enumerate}
  \item Estime quantas iterações são necessárias para o Método da
    Bissecção achar a raiz da função $f(x)=\ln x$ em $[0.5, 3.5]$ com
    precisão $\varepsilon < 10^{-2}$
  \item Aplique o Método de Newton com valor inicial $x_0 = 2$ até esta
    precisão.
  \item Compare o número de iterações necessário.
  \item (Perspectiva) Qual é a raiz exata desta função no intervalo
    acima?
  \end{enumerate}
 

\item O número $\pi$ pode ser aproximado usando o método de Newton
  usando a função $f(x) = \cos x -1$ e o valor inicial $x_0 =
  3$. Encontre uma aproximação com precisão de $\varepsilon < 10^{-3}$

\item (Conjugação de métodos) Quando não se tem um bom ponto de
  partida $x_0$ para se aplicar o Método de Newton, podemos usar
  algumas iterações do Método da Bissecção para obtê-la.

\end{enumerate}

\end{document}
