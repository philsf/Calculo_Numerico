\everymath{\displaystyle}
%\documentclass[pdftex,a4paper]{article}
\documentclass[a4paper]{article}
%%classes: article, report, book, proc, amsproc

%%%%%%%%%%%%%%%%%%%%%%%%
%% Misc
% para acertar os acentos
\usepackage[brazilian]{babel} 
%\usepackage[portuguese]{babel} 
% \usepackage[english]{babel}
% \usepackage[T1]{fontenc}
% \usepackage[latin1]{inputenc}
\usepackage[utf8]{inputenc}
\usepackage{indentfirst}
\usepackage{fullpage}
% \usepackage{graphicx} %See PDF section
\usepackage{multicol}
\setlength{\columnseprule}{0.5pt}
\setlength{\columnsep}{20pt}
%%%%%%%%%%%%%%%%%%%%%%%%
%%%%%%%%%%%%%%%%%%%%%%%%
%% PDF support

\usepackage[pdftex]{color,graphicx}
% %% Hyper-refs
\usepackage[pdftex]{hyperref} % for printing
% \usepackage[pdftex,bookmarks,colorlinks]{hyperref} % for screen

%% \newif\ifPDF
%% \ifx\pdfoutput\undefined\PDFfalse
%% \else\ifnum\pdfoutput > 0\PDFtrue
%%      \else\PDFfalse
%%      \fi
%% \fi

%% \ifPDF
%%   \usepackage[T1]{fontenc}
%%   \usepackage{aeguill}
%%   \usepackage[pdftex]{graphicx,color}
%%   \usepackage[pdftex]{hyperref}
%% \else
%%   \usepackage[T1]{fontenc}
%%   \usepackage[dvips]{graphicx}
%%   \usepackage[dvips]{hyperref}
%% \fi

%%%%%%%%%%%%%%%%%%%%%%%%


%%%%%%%%%%%%%%%%%%%%%%%%
%% Math
\usepackage{amsmath,amsfonts,amssymb}
% para usar R de Real do jeito que o povo gosta
\usepackage{amsfonts} % \mathbb
% para usar as letras frescas como L de Espaco das Transf Lineares
% \usepackage{mathrsfs} % \mathscr

% Oferecer seno e tangente em pt, com os comandos usuais.
\providecommand{\sin}{} \renewcommand{\sin}{\hspace{2pt}\mathrm{sen}}
\providecommand{\tan}{} \renewcommand{\tan}{\hspace{2pt}\mathrm{tg}}

% dt of integrals = \ud t
\newcommand{\ud}{\mathrm{\ d}}
%%%%%%%%%%%%%%%%%%%%%%%%



\begin{document}

%%%%%%%%%%%%%%%%%%%%%%%%
%% Título e cabeçalho
%\noindent\parbox[c]{.15\textwidth}{\includegraphics[width=.15\textwidth]{logo}}\hfill
\parbox[c]{.825\textwidth}{\raggedright%
  \sffamily {\LARGE

Cálculo Numérico: Lista de Eliminação de Gauss

\par\bigskip}
{Prof: Felipe Figueiredo\par}
{\url{http://sites.google.com/site/proffelipefigueiredo}\par}
}

Versão: \verb|20150519|

%%%%%%%%%%%%%%%%%%%%%%%%


%%%%%%%%%%%%%%%%%%%%%%%%
\section{Formulário}

Em cada etapa $i$, temos:

Pivô: $a_{ii}$

Multiplicadores:

$m_{ji} = \frac{a_{ji}}{a_{ii}}$, onde $j>i$.

Operações com Linhas:

$L_j = L_j - m_{ji}L_i$, onde $j>i$, $L_j$ é a linha a ser eliminada,
e $L_i$ é a linha do pivô da etapa $i$.

\section{Exercícios}

\begin{enumerate}
\item Para cada matriz de coeficientes $A$ e cada vetor $b$ abaixo,
  resolva o sistema linear $Ax=b$ usando o Método da Eliminação de
  Gauss:
  \begin{enumerate}
  \item % A=[ 3,4; 2,3], b=[7;5], [L,U,P]=lu(A), y=L\b, x=U\y
    $A = \begin{bmatrix}
        3 & 4\\
        2 & 3\\
      \end{bmatrix},
      b= \begin{bmatrix}
        7\\
        5\\
      \end{bmatrix}$

  \item % A=[5,7,11 ; -1,0,2; -1,0,3], b=[10;12;0], [L,U,P]=lu(A), y=L\b, x=U\y
    $A = \begin{bmatrix}
      5 & 7 & 11\\
      -1 & 0 & 2\\
      -1 & 0 & 3\\
      \end{bmatrix},
      b= \begin{bmatrix}
        10\\
        12\\
        0\\
      \end{bmatrix}$

  \item % A=[22,-44,6; -5,7,-1; -11,22,-3], b=[1;2;-2], [L,U,P]=lu(A), y=L\b, x=U\y
    $A = \begin{bmatrix}
      22 & -44 & 6\\
      -5 & 7 & -1\\
      -11 & 22 & -3\\
      \end{bmatrix},
      b= \begin{bmatrix}
        1\\
        2\\
        -2\\
      \end{bmatrix}$

  \item % A = [-5.0,0.2,-0.1,4.0;1.1,1.2,1.3,0.0;0.9,-1.0,3.7,0.0;-2.3,0.2,-0.1,0.0], b=[0.0;0.1;-2.5;-5.2], [L,U,P]=lu(A), y=L\b, x=U\y
    $A = \begin{bmatrix}
      -5.0 & 0.2 & -0.1 & 4.0\\
      1.1 & 1.2 & 1.3 & 0.0\\
      0.9 & -1.0 & 3.7 & 0.0\\
      -2.3 & 0.2 & -0.1 & 0.0\\
      \end{bmatrix},
      b= \begin{bmatrix}
        0.0\\
        0.1\\
        -2.5\\
        -5.2\\
      \end{bmatrix}$

  \item % A = [-3.76,2.00,1.71;1.00,-2.04,0.00; 0.00,0.25,0.37], b= [0.13; 0.45; 0.50], [L,U,P]=lu(A), y=L\b, x=U\y
    $A = \begin{bmatrix}
      -3.76 & 2.00 & 1.71\\
      1.00 & -2.04 & 0.00\\
      0.00 & 0.25 & 0.37\\
      \end{bmatrix},
      b= \begin{bmatrix}
        0.13\\
        0.45\\
        0.50\\
      \end{bmatrix}$

  \item % A = [-1.4,0.0,4.0,-2.0;-0.7,2.2,2.6,-1.3;1.0,2.0,3.0,-1.5;0.0,0.0,-0.2,0.1], b=[2.0;3.0;1.0;4.0], [L,U,P]=lu(A), y=L\b, x=U\y
    $A = \begin{bmatrix}
      -1.4 & 0.0 & 4.0 & -2.0\\
      -0.7 & 2.2 & 2.6 & -1.3\\
      1.0 & 2.0 & 3.0 & -1.5\\
      0.0 & 0.0 & -0.2 & 0.1\\
    \end{bmatrix},
    b= \begin{bmatrix}
      2.0\\
      3.0\\
      1.0\\
      4.0\\
    \end{bmatrix}$

  % \item % A = [-1.4,0.0,4.0,-2.0;-0.7,2.2,2.6,-1.3;1.0,2.0,3.0,-1.5;0.0,0.0,-0.2,0.1 ], b=[2.0;3.0;1.0;0.0], [L,U,P]=lu(A), y=L\b, x=U\y
  %   $A = \begin{bmatrix}
  %     -1.4 & 0.0 & 4.0 & -2.0\\
  %     -0.7 & 2.2 & 2.6 & -1.3\\
  %     1.0 & 2.0 & 3.0 & -1.5\\
  %     0.0 & 0.0 & -0.2 & 0.1\\
  %     \end{bmatrix},
  %     b= \begin{bmatrix}
  %       2.0\\
  %       3.0\\
  %       1.0\\
  %       0.0\\
  %     \end{bmatrix}$

  % \item % 
  %   $A = \begin{bmatrix}
  %     \\
  %     \end{bmatrix},
  %     b= \begin{bmatrix}
  %       \\
  %     \end{bmatrix}$

  \end{enumerate}

\end{enumerate}


\end{document}
