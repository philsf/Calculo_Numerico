\everymath{\displaystyle}
%\documentclass[pdftex,a4paper]{article}
\documentclass[a4paper]{article}
%%classes: article, report, book, proc, amsproc

%%%%%%%%%%%%%%%%%%%%%%%%
%% Misc
% para acertar os acentos
\usepackage[brazilian]{babel} 
%\usepackage[portuguese]{babel} 
% \usepackage[english]{babel}
% \usepackage[T1]{fontenc}
% \usepackage[latin1]{inputenc}
\usepackage[utf8]{inputenc}
\usepackage{indentfirst}
\usepackage{fullpage}
% \usepackage{graphicx} %See PDF section
\usepackage{multicol}
\setlength{\columnseprule}{0.5pt}
\setlength{\columnsep}{20pt}
%%%%%%%%%%%%%%%%%%%%%%%%
%%%%%%%%%%%%%%%%%%%%%%%%
%% PDF support

\usepackage[pdftex]{color,graphicx}
% %% Hyper-refs
\usepackage[pdftex]{hyperref} % for printing
% \usepackage[pdftex,bookmarks,colorlinks]{hyperref} % for screen

%% \newif\ifPDF
%% \ifx\pdfoutput\undefined\PDFfalse
%% \else\ifnum\pdfoutput > 0\PDFtrue
%%      \else\PDFfalse
%%      \fi
%% \fi

%% \ifPDF
%%   \usepackage[T1]{fontenc}
%%   \usepackage{aeguill}
%%   \usepackage[pdftex]{graphicx,color}
%%   \usepackage[pdftex]{hyperref}
%% \else
%%   \usepackage[T1]{fontenc}
%%   \usepackage[dvips]{graphicx}
%%   \usepackage[dvips]{hyperref}
%% \fi

%%%%%%%%%%%%%%%%%%%%%%%%


%%%%%%%%%%%%%%%%%%%%%%%%
%% Math
\usepackage{amsmath,amsfonts,amssymb}
% para usar R de Real do jeito que o povo gosta
\usepackage{amsfonts} % \mathbb
% para usar as letras frescas como L de Espaco das Transf Lineares
% \usepackage{mathrsfs} % \mathscr

% Oferecer seno e tangente em pt, com os comandos usuais.
\providecommand{\sin}{} \renewcommand{\sin}{\hspace{2pt}\mathrm{sen}}
\providecommand{\tan}{} \renewcommand{\tan}{\hspace{2pt}\mathrm{tg}}

% dt of integrals = \ud t
\newcommand{\ud}{\mathrm{\ d}}
%%%%%%%%%%%%%%%%%%%%%%%%



\begin{document}

%%%%%%%%%%%%%%%%%%%%%%%%
%% Título e cabeçalho
%\noindent\parbox[c]{.15\textwidth}{\includegraphics[width=.15\textwidth]{logo}}\hfill
\parbox[c]{.825\textwidth}{\raggedright%
  \sffamily {\LARGE

Cálculo Numérico: Gabarito de Método de Newton

\par\bigskip}
{Prof: Felipe Figueiredo\par}
{\url{http://sites.google.com/site/proffelipefigueiredo}\par}
}

Versão: \verb|20150519.2|

%%%%%%%%%%%%%%%%%%%%%%%%


%%%%%%%%%%%%%%%%%%%%%%%%
\section{}
\section{}

\begin{enumerate}
\item % Encontre uma aproximação para a raiz das funções abaixo, com
  % cada ponto inicial dado. Use o método de Newton até atingir a
  % precisão de $\varepsilon<10^{-2}$ ou $k=4$ passos, o que ocorrer
  % primeiro. Identifique na sua resposta a sequência $x_k$ obtida, e
  % use o último $x_k$ como resposta aproximada $\bar{x}$:

  \begin{enumerate}
  % \item $\bar{x}=0, x_1=0, x_2=0, \varepsilon=0$ % $f(x) = x$, com $x_0 = 1.5$
  % \item $\bar{x}=0, x_1=0, x_2=0, \varepsilon=$ % $f(x) = x$, com $x_0 = 10$
  \item $\bar{x}=2.00455764, x_1=2.9, x_2=2.13965517, x_3=2.00455764, x_4=, \varepsilon=0.13509752$ % $f(x) = x^2 - 4$, com $x_0 = 5$
  \item $\bar{x}=-0.59259259, x_1=-2, x_2=-1.33333333, x_3=-0.88888888, x_4=-0.59259259, \varepsilon=0.44444444$ % $f(x) = x^3$, com $x_0 = -3$
  \item $\bar{x}=1.51677444, x_1=4.05633802, x_2=2.78897669, x_3=1.98704642, x_4=1.51677444, \varepsilon=0.47027$ % $f(x) = x^3-1.5x$, com $x_0 = 6$
  \item $\bar{x}=0.00129550, x_1=0.57619047, x_2=0.21063156, x_3=0.03604670, x_4=0.00129550, \varepsilon=0.03535119$ % $f(x) = x e^x$, com $x_0 = 1.1$
  \item $\bar{x}=2.92 \times 10^{-13},x_1=-0.55740772, x_2=0.06593645, x_3=-0.00009752, x_4=2.92 \times 10^{-13}, \varepsilon=$ % $f(x) = \sin x$, com $x_0 = 1$
  \end{enumerate}

\item 
  \begin{enumerate}
  % \item % $x=0$ %$f(x) = x$, em $[-1, 3]$
  % \item % $x=0$ %$f(x) = x$, em $[-1, 4]$
  \item % $x=2$ %$f(x) = x^2 - 4$, em $[1.5, 3]$
  \item % $x=0$ %$f(x) = x^3$, em $[-0.5, 1]$
  \item % $x=\sqrt{1.5}$ %$f(x) = x^3-1.5x$, em $[1, 2.5]$
  \item % $x=0$ %$f(x) = x e^x$, em $[-0.5, 1]$
  \item % $x=0$ %$f(x) = \sin x$, em $\left[\frac{-\pi}{2},\frac{\pi}{2}\right]$
  % \item $x=$ %$f(x) = $, em $[]$
  % \item $x=$ %$f(x) = $, em $[]$
  \end{enumerate}

\item % Determine o erro absoluto e o erro relativo da aproximação
  % $\bar{x}$ encontrada em cada item do exercício 1, considerando que
  % as soluções exatas são:
  \begin{enumerate}
  \item % $x=0$ %$f(x) = x$, em $[-1, 3]$
  \item % $x=0$ %$f(x) = x$, em $[-1, 4]$
  \item % $x=2$ %$f(x) = x^2 - 4$, em $[1.5, 3]$
  \item % $x=0$ %$f(x) = x^3$, em $[-0.5, 1]$
  \item % $x=\sqrt{1.5}$ %$f(x) = x^3-1.5x$, em $[1, 2.5]$
  \item % $x=0$ %$f(x) = x e^x$, em $[-0.5, 1]$
  \item % $x=0$ %$f(x) = \sin x$, em $\left[\frac{-\pi}{2},\frac{\pi}{2}\right]$
  % \item $x=$ %$f(x) = $, em $[]$
  % \item $x=$ %$f(x) = $, em $[]$
  \end{enumerate}

\section{}

\item
  \begin{enumerate}
  \item % Estime quantas iterações são necessárias para o Método da
    % Bissecção achar a raiz da função $f(x)=\ln x$ em $[0.5, 3.5]$ com
    % precisão $\varepsilon < 10^{-2}$
  \item % Aplique o Método de Newton com valor inicial $x_0 = 2$ até esta
    % precisão.
  \item % Compare o número de iterações necessário.
  \item % (Perspectiva) Qual é a raiz exata desta função no intervalo
    % acima?
  \end{enumerate}
 

\item $\bar{x} = 3.14149311, x_1=3.14079632, x_2=3.14119449,
  x_3=3.14139367, x_4=3.14149311,
  \varepsilon=|x_4-x_3|=0.00009954$ % O
  % número $\pi$ pode ser aproximado usando o método de Newton usando a
  % função $f(x) = \cos x +1$ e o valor inicial $x_0 = 3.14$. Encontre
  % uma aproximação com precisão de $\varepsilon < 10^{-4}$

\item % (Conjugação de métodos) Quando não se tem um bom ponto de
  % partida $x_0$ para se aplicar o Método de Newton, podemos usar
  % algumas iterações do Método da Bissecção para obtê-la. Considere a
  % função $f(x) = e^{2x} (x^3-15x^2+1)$. Vamos encontrar uma
  % aproximação para a raiz desta função contida no intervalo $[-1,0.1]$
  % com precisão de $\varepsilon<0.001$.
  \begin{enumerate}
  \item $f'(x) = 2e^{2x}(x^3-15x^2+1) + e^{2x}(3x^2-30x)$ %Qual é a derivada desta função?
  \item $a=-1$, $b=0.1$ portanto $f(a) = -2.03002924$ e $f(b) =
    1.03941374$
    % Verifique que a função troca de sinais no intervalo $[-1,0.1]$,
    % e portanto o método da Bissecção pode ser aplicada
    % para encontrar uma raiz aproximada para ela.
  \item $x_2 = -0.175$  
    % Aplique o Método da Bissecção por duas iterações, e encontre
    % $x_2$ no intervalo $[-1,0.1]$. Verifique que para este método,
    % você precisaria de 12 iterações para atingir a precisão de
    % $0.001$
  \item $\bar{x} = x_2 = -0.256023160108161823$

\begin{tabular}{c|c|c|c|c}
  $k$ & $x_k$ & $f(x_k)$ & $f'(x_k)$ & $\varepsilon = |f(x_k)|$\\
  \hline
  0 & -0.175 & 0.37719531077 & 4.5187463108 & 0.37719531077\\
  \hline
  1 & -0.25847344259 & -0.011566602376 & 4.7205179167 & 0.011566602376\\
  \hline
  2 & -0.25602316010 & 2.112113888E-7 & 4.7206374662 & 2.112113888E-7\\
    \end{tabular}
    % Use o valor do item anterior como valor inicial $x_0$ do Método
    % de Newton, e encontre uma aproximação com precisão de
    % $0.001$. (Obs: se você utilizar o critério de parada
    % $\varepsilon =
    % |f(x_n)|$, você precisará de 2 iterações. Se você utilizar o
    % critério $\varepsilon = |x_n - x_{n-1}|$ precisará de 3
    % iterações.)
  \end{enumerate}


\end{enumerate}

\end{document}
