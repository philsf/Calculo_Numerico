\everymath{\displaystyle}
%\documentclass[pdftex,a4paper]{article}
\documentclass[a4paper]{article}
%%classes: article, report, book, proc, amsproc

%%%%%%%%%%%%%%%%%%%%%%%%
%% Misc
% para acertar os acentos
\usepackage[brazilian]{babel} 
%\usepackage[portuguese]{babel} 
% \usepackage[english]{babel}
% \usepackage[T1]{fontenc}
% \usepackage[latin1]{inputenc}
\usepackage[utf8]{inputenc}
\usepackage{indentfirst}
\usepackage{fullpage}
% \usepackage{graphicx} %See PDF section
\usepackage{multicol}
\setlength{\columnseprule}{0.5pt}
\setlength{\columnsep}{20pt}
%%%%%%%%%%%%%%%%%%%%%%%%
%%%%%%%%%%%%%%%%%%%%%%%%
%% PDF support

\usepackage[pdftex]{color,graphicx}
% %% Hyper-refs
\usepackage[pdftex]{hyperref} % for printing
% \usepackage[pdftex,bookmarks,colorlinks]{hyperref} % for screen

%% \newif\ifPDF
%% \ifx\pdfoutput\undefined\PDFfalse
%% \else\ifnum\pdfoutput > 0\PDFtrue
%%      \else\PDFfalse
%%      \fi
%% \fi

%% \ifPDF
%%   \usepackage[T1]{fontenc}
%%   \usepackage{aeguill}
%%   \usepackage[pdftex]{graphicx,color}
%%   \usepackage[pdftex]{hyperref}
%% \else
%%   \usepackage[T1]{fontenc}
%%   \usepackage[dvips]{graphicx}
%%   \usepackage[dvips]{hyperref}
%% \fi

%%%%%%%%%%%%%%%%%%%%%%%%


%%%%%%%%%%%%%%%%%%%%%%%%
%% Math
\usepackage{amsmath,amsfonts,amssymb}
% para usar R de Real do jeito que o povo gosta
\usepackage{amsfonts} % \mathbb
% para usar as letras frescas como L de Espaco das Transf Lineares
% \usepackage{mathrsfs} % \mathscr

% Oferecer seno e tangente em pt, com os comandos usuais.
\providecommand{\sin}{} \renewcommand{\sin}{\hspace{2pt}\mathrm{sen}}
\providecommand{\tan}{} \renewcommand{\tan}{\hspace{2pt}\mathrm{tg}}

% dt of integrals = \ud t
\newcommand{\ud}{\mathrm{\ d}}
%%%%%%%%%%%%%%%%%%%%%%%%



\begin{document}

%%%%%%%%%%%%%%%%%%%%%%%%
%% Título e cabeçalho
%\noindent\parbox[c]{.15\textwidth}{\includegraphics[width=.15\textwidth]{logo}}\hfill
\parbox[c]{.825\textwidth}{\raggedright%
  \sffamily {\LARGE

Cálculo Numérico: Gabarito das Etapas 3 e 4 da ATPS

\par\bigskip}
{Prof: Felipe Figueiredo\par}
{\url{http://sites.google.com/site/proffelipefigueiredo}\par}
}

Versão: \verb|20150518|

%%%%%%%%%%%%%%%%%%%%%%%%

Este gabarito foi feito utilizando o software Octave, um clone
gratuito do Matlab.  Obs: Este gabarito \emph{não} foi feito
utilizando três algarismos significativos, portanto deve ser utilizado
com cautela e apenas para conferir os desenvolvimentos necessários
para a elaboração do Trabalho 2.

Não copie as respostas aqui contidas para seu trabalho. Lembre-se,
respostas sem justificativas e cálculos não serão consideradas!

%%%%%%%%%%%%%%%%%%%%%%%%
\section*{Etapa 3}

\subsection*{Passo 2}

Matriz dos coeficientes do sistema:

\begin{verbatim}
A =
    1    1    1
   10   -8    0
    8    0   -3
\end{verbatim}

Determinante:

\begin{verbatim}
det(A)
ans =  118
\end{verbatim}

Matriz inversa ($A^{-1}$):

\begin{verbatim}
inv(A)
ans =
   0.203390   0.025424   0.067797
   0.254237  -0.093220   0.084746
   0.542373   0.067797  -0.152542
\end{verbatim}

Solução do sistema $Ax=b$:

\begin{verbatim}
x =
    9.7881
    4.1102
  -13.8983
\end{verbatim}



\section*{Etapa 4}

\subsection*{Passo 2}

\subsubsection*{Desafio A}

Matriz dos coeficientes:

\begin{verbatim}
A =
   2.00000   1.00000   3.00000   0.00000
   2.00000   2.00000   5.00000   1.00000
   2.00000   1.00000   4.00000   0.00000
   1.00000   1.00000   3.50000   2.50000
\end{verbatim}

Matriz L:

\begin{verbatim}
L =
   1.00000   0.00000   0.00000   0.00000
   1.00000   1.00000   0.00000   0.00000
   1.00000   0.00000   1.00000   0.00000
   0.50000   0.50000   1.00000   1.00000
\end{verbatim}

Matriz U:

\begin{verbatim}
U =
   2   1   3   0
   0   1   2   1
   0   0   1   0
   0   0   0   2
\end{verbatim}

\subsubsection*{Desafio B}

(a)

\begin{verbatim}
A =
   4  -1   1
   2   5   2
   1   2   4

b =
    8
    3
   11

x =
   1
  -1
   3
\end{verbatim}

(b)
\begin{verbatim}
A =
   1   1   0   1
   2   1  -1   1
  -1   1   3  -1
   3  -1  -1   3

b =
   2
   1
   4
  -3

x =
  -0.40000
   2.10000
   0.60000
   0.30000
\end{verbatim}

Determinante da matriz do item (b):

\begin{verbatim}
det(A)
ans = -10
\end{verbatim}

\end{document}
