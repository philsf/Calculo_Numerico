\everymath{\displaystyle}
%\documentclass[pdftex,a4paper]{article}
\documentclass[a4paper]{article}
%%classes: article, report, book, proc, amsproc

%%%%%%%%%%%%%%%%%%%%%%%%
%% Misc
% para acertar os acentos
\usepackage[brazilian]{babel} 
%\usepackage[portuguese]{babel} 
% \usepackage[english]{babel}
% \usepackage[T1]{fontenc}
% \usepackage[latin1]{inputenc}
\usepackage[utf8]{inputenc}
\usepackage{indentfirst}
\usepackage{fullpage}
% \usepackage{graphicx} %See PDF section
\usepackage{multicol}
\setlength{\columnseprule}{0.5pt}
\setlength{\columnsep}{20pt}
%%%%%%%%%%%%%%%%%%%%%%%%
%%%%%%%%%%%%%%%%%%%%%%%%
%% PDF support

\usepackage[pdftex]{color,graphicx}
% %% Hyper-refs
\usepackage[pdftex]{hyperref} % for printing
% \usepackage[pdftex,bookmarks,colorlinks]{hyperref} % for screen

%% \newif\ifPDF
%% \ifx\pdfoutput\undefined\PDFfalse
%% \else\ifnum\pdfoutput > 0\PDFtrue
%%      \else\PDFfalse
%%      \fi
%% \fi

%% \ifPDF
%%   \usepackage[T1]{fontenc}
%%   \usepackage{aeguill}
%%   \usepackage[pdftex]{graphicx,color}
%%   \usepackage[pdftex]{hyperref}
%% \else
%%   \usepackage[T1]{fontenc}
%%   \usepackage[dvips]{graphicx}
%%   \usepackage[dvips]{hyperref}
%% \fi

%%%%%%%%%%%%%%%%%%%%%%%%


%%%%%%%%%%%%%%%%%%%%%%%%
%% Math
\usepackage{amsmath,amsfonts,amssymb}
% para usar R de Real do jeito que o povo gosta
\usepackage{amsfonts} % \mathbb
% para usar as letras frescas como L de Espaco das Transf Lineares
% \usepackage{mathrsfs} % \mathscr

% Oferecer seno e tangente em pt, com os comandos usuais.
\providecommand{\sin}{} \renewcommand{\sin}{\hspace{2pt}\mathrm{sen}}
\providecommand{\tan}{} \renewcommand{\tan}{\hspace{2pt}\mathrm{tg}}

% dt of integrals = \ud t
\newcommand{\ud}{\mathrm{\ d}}
%%%%%%%%%%%%%%%%%%%%%%%%



\begin{document}

%%%%%%%%%%%%%%%%%%%%%%%%
%% Título e cabeçalho
%\noindent\parbox[c]{.15\textwidth}{\includegraphics[width=.15\textwidth]{logo}}\hfill
\parbox[c]{.825\textwidth}{\raggedright%
  \sffamily {\LARGE

Cálculo Numérico: Trabalho 0

\par\bigskip}
{Prof: Felipe Figueiredo\par}
{\url{http://sites.google.com/site/proffelipefigueiredo}}

\vspace{1cm}
}
%%%%%%%%%%%%%%%%%%%%%%%%


%%%%%%%%%%%%%%%%%%%%%%%%
\section{Objetivo}
O objetivo principal deste trabalho é proporcionar aos alunos uma
primeira experiência com a interpretação e implementação de uma norma
técnica. O objetivo secundário é ambientar os alunos com o método de
organização de envio de trabalho por email para o professor.

\section{Valor}
O trabalho valerá $0.5$pt na P1, bastando que a entrega seja válida.

\section{Conteúdo}
O grupo deve fazer a seguinte conversão de base:
 
\begin{enumerate}
\item Converter o número $11111011111_2$ da base binária para a base
  decimal Todos os cálculos necessários devem ser apresentados no
  trabalho.
\end{enumerate}

\section{Entrega}
A interpretação e implementação de normas técnicas é parte importante
do trabalho do engenheiro, seja para a implementação de um processo em
uma linha de produção em fábrica, produção de peças atendendo às
especificações técnicas necessárias ao funcionamento de máquinas ou
mesmo a correta mistura na composição de materiais de construção ou
metalúrgicos. A falha na implementação de todos os detalhes de uma
especificação pode gerar prejuízos à empresa ou cliente, o que pode
inviabilizar a manutenção do emprego do engenheiro responsável.

Devido à grande quantidade de emails recebidos de alunos, é necessário
que os alunos se identifiquem ao enviar o trabalho. Para que eu possa
organizar os trabalhos recebidos, preciso que os alunos se adequem a
um padrão de identificação. A norma técnica aqui contida descreve o
processo de organização para entrega de trabalhos acadêmicos para a
disciplina.

O trabalho 0 deve ser entregue por email para o endereço
\url{prof.felipefigueiredo@gmail.com} até a data determinada no
endereço
\url{https://sites.google.com/site/proffelipefigueiredo/anhanguera/engenharia}.
 
Enviar os trabalhos por email até a meia noite da data marcada. Não
serão considerados trabalhos enviados posteriormente.

\subsection{Especificação}

\begin{itemize}
\item Identifique no assunto do email o curso, turno (manhã ou noite),
  e a turma.
\item Identifique no corpo do email os nomes em ordem alfabética e RAs
  dos alunos.
\item O trabalho deve ser anexado ao email no formato PDF.
\end{itemize}

\subsection{Exemplo}

O grupo é do curso de Engenharia Civil, no turno da Manhã, do quinto
período turma B. Os campos do email devem ser preenchidos como segue:

\begin{tabular}{|l|}
  \hline
  Assunto: Trabalho 2 Engenharia Civil 5B Manhã\\
  \hline
  Fulano da Silva RA 0000001 \\
  Beltrano da Silva RA 0000002 \\
  \ldots \\
  Ciclano da Silva RA 0000010  \\
  \hline
\end{tabular}
 
É dever do aluno zelar para que seu trabalho esteja propriamente
identificado, para que seja corrigido. Trabalhos que não aderirem às
especificações acima serão desconsiderados.
 
Trabalhos textuais e relatórios devem ser convertidos para o formato
PDF. Não deixe de ler as regras para trabalhos feitos em casa.

\subsection{Observações}

\begin{enumerate}
\item A sala de aula não identifica a turma! A sala de aula pode ser
  trocada após o início das aulas, ao passo que a turma é a informação
  cadastrada no sistema, na pauta de presença e ata de provas. Assim,
  a turma será sempre identificada pelo período e uma letra (e.g.,
  turma B do quinto período: 5B, conforme o exemplo acima).
\item O objetivo de qualquer trabalho em grupo é proporcionar aos
  alunos o treinamento do trabalho em equipe, e as estratégias de
  negociação e comunicação. O objetivo NÃO É reduzir a carga de
  estudo, dividindo tarefas para cada componente do grupo. Cada
  componente do grupo deve compreender e ser capaz de explicar
  oralmente qualquer etapa do trabalho.

\end{enumerate}

 


\end{document}
