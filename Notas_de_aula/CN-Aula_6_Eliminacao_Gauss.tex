\everymath{\displaystyle}
%\documentclass[pdftex,a4paper]{article}
\documentclass[a4paper]{article}
%%classes: article, report, book, proc, amsproc

%%%%%%%%%%%%%%%%%%%%%%%%
%% Misc
% para acertar os acentos
\usepackage[brazilian]{babel} 
%\usepackage[portuguese]{babel} 
% \usepackage[english]{babel}
% \usepackage[T1]{fontenc}
% \usepackage[latin1]{inputenc}
\usepackage[utf8]{inputenc}
\usepackage{indentfirst}
\usepackage{fullpage}
% \usepackage{graphicx} %See PDF section
\usepackage{multicol}
\setlength{\columnseprule}{0.5pt}
\setlength{\columnsep}{20pt}
%%%%%%%%%%%%%%%%%%%%%%%%
%%%%%%%%%%%%%%%%%%%%%%%%
%% PDF support

\usepackage[pdftex]{color,graphicx}
% %% Hyper-refs
\usepackage[pdftex]{hyperref} % for printing
% \usepackage[pdftex,bookmarks,colorlinks]{hyperref} % for screen

%% \newif\ifPDF
%% \ifx\pdfoutput\undefined\PDFfalse
%% \else\ifnum\pdfoutput > 0\PDFtrue
%%      \else\PDFfalse
%%      \fi
%% \fi

%% \ifPDF
%%   \usepackage[T1]{fontenc}
%%   \usepackage{aeguill}
%%   \usepackage[pdftex]{graphicx,color}
%%   \usepackage[pdftex]{hyperref}
%% \else
%%   \usepackage[T1]{fontenc}
%%   \usepackage[dvips]{graphicx}
%%   \usepackage[dvips]{hyperref}
%% \fi

%%%%%%%%%%%%%%%%%%%%%%%%


%%%%%%%%%%%%%%%%%%%%%%%%
%% Math
\usepackage{amsmath,amsfonts,amssymb}
% para usar R de Real do jeito que o povo gosta
\usepackage{amsfonts} % \mathbb
% para usar as letras frescas como L de Espaco das Transf Lineares
% \usepackage{mathrsfs} % \mathscr

% Oferecer seno e tangente em pt, com os comandos usuais.
\providecommand{\sin}{} \renewcommand{\sin}{\hspace{2pt}\mathrm{sen}}
\providecommand{\tan}{} \renewcommand{\tan}{\hspace{2pt}\mathrm{tg}}

% dt of integrals = \ud t
\newcommand{\ud}{\mathrm{\ d}}
%%%%%%%%%%%%%%%%%%%%%%%%



\begin{document}

%%%%%%%%%%%%%%%%%%%%%%%%
%% Título e cabeçalho
%\noindent\parbox[c]{.15\textwidth}{\includegraphics[width=.15\textwidth]{logo}}\hfill
\parbox[c]{.825\textwidth}{\raggedright%
  \sffamily {\LARGE

Cálculo Numérico: Notas de Aula: Eliminação de Gauss

\par\bigskip}
{Prof: Felipe Figueiredo\par}
{\url{http://sites.google.com/site/proffelipefigueiredo}\par}
}

Versão: \verb|20150503|

%%%%%%%%%%%%%%%%%%%%%%%%


%%%%%%%%%%%%%%%%%%%%%%%%


\section*{3.2.2 Método da Eliminação de Gauss}

\subsection*{Resolução de um Sistema Triangular Superior}

Aula da semana anterior (revisão de Álgebra Linear)

\subsection*{Descrição do método da Eliminação de Gauss}

\subsubsection*{Teorema 1}

Seja $Ax=b$ um sistema linear. As seguintes operações resultam em um
sistema $\tilde{A}x=\tilde{b}$ equivalente (mesma solução $x$):

\begin{enumerate}
\item Trocar duas equações
\item multiplicar uma equação por uma constante não-nula
\item adicionar um múltiplo de uma equação a uma outra equação
\end{enumerate}

Como resolver o sistema $Ax=b$?

Para cada etapa $i$, o pivô $a_{ii} \ne 0$

\subsubsection*{Etapa 1}

A eliminação da variável $x_1$ nas equações 2 até $n$ é feita usando
um multiplicador $m_{i1}$ que depende do termo $a_{i1}$ e do pivô
$a_{11}$.

Ex: 

$m_{21} = \frac{a_{21}}{a_{11}}$, $m_{31} = \frac{a_{31}}{a_{11}}$, etc.

Após descobrir cada multiplicador, subtrair as linhas para eliminar os
termos abaixo do pivô:

$L_2 = L_2 - m_{21}L_1$

$L_3 = L_3 - m_{31}L_1$

Assim eliminamos todos os elementos abaixo do pivô da primeira coluna.

\subsubsection*{Etapa 2}

Para eliminarmos os elementos abaixo do pivô da segunda coluna,
procedemos da mesma maneira como acima.

O pivô da segunda linha será o {\bf novo} elemento $a_{22}$.

O multiplicador para a terceira linha dependerá do novo elemento
$a_{32}$ e o novo pivô $a_{22}$:

$m_{32} = \frac{a_{32}}{a_{22}}$

E a operação para eliminar o elemento da terceira linha será:

$L_3 = L_3 - m_{32}L_2$

\subsubsection*{Etapa (n-1)}

Analogamente, procedemos em cada coluna, até a última, eliminando
todos os termos abaixo de cada pivô, sempre assumindo que o pivô é
diferente de 0.

\subsubsection*{Exemplo}
Seja o Sistema linear:

\begin{displaymath}
  \left\{
  \begin{array}{ccccc}
    3x_1 &+ 2x_2 &+ 4x_3 &=&1\\
    x_1 &+ x_2 &+ 2x_3 &=& 2\\
    4x_1 &+ 3x_2 &- 2x_3 &=& 3\\
  \end{array}
\right.
\end{displaymath}

Matriz aumentada:

\begin{displaymath}
  A^{(0)}|b^{(0)} = \begin{bmatrix}
    3 & 2 & 4 &|& 1\\
    1 & 1 & 2 &|& 2\\
    4 & 3 & -2 &|& 3\\
  \end{bmatrix}
\end{displaymath}

Etapa 1: eliminar $x_1$ das equações 2 e 3:

Pivô: $a_{11}=3$

$m_{21} = \frac{1}{3}$

$m_{31} = \frac{4}{3}$

$L_2 = L_2 - m_{21}L_1$

$L_3 = L_3 - m_{31}L_1$

\begin{displaymath}
  A^{(1)}|b^{(1)} = \begin{bmatrix}
    3 & 2 & 4 &|& 1\\
    0 & \frac{1}{3} & \frac{2}{3} &|& \frac{5}{3}\\
    0 & \frac{1}{3} & \frac{-22}{3} &|& \frac{5}{3}\\
  \end{bmatrix}
\end{displaymath}


Etapa 2: eliminar $x_2$ da equação 3:

Pivô: $a_{22} = \frac{1}{3}$

$m_{32} = \frac{1/3}{1/3} = 1$

$L_3 = L_3 - m_{32}L_2$

\begin{displaymath}
  A^{(2)}|b^{(2)} = \begin{bmatrix}
    3 & 2 & 4 &|& 1\\
    0 & \frac{1}{3} & \frac{2}{3} &|& \frac{5}{3}\\
    0 & 0 & -8 &|& 0\\
  \end{bmatrix}
\end{displaymath}

Assim, resolver o sistema $Ax=b$ é equivalente a resolver
$A^{(2)}x=b^{(2)}$:

\begin{displaymath}
  \left\{
  \begin{array}{ccccc}
    3x_1 &+ 2x_2 &+ 4x_3 &=&1\\
     & \frac{1}{3} x_2 &+ \frac{2}{3}x_3 &=& \frac{5}{3}\\
     & &- 8x_3 &=& 0\\
  \end{array}
\right.
\end{displaymath}

\subsubsection*{Exercício}

Resolva o sistema linear utilizando o método de Eliminação de Gauss:

\begin{displaymath}
  \left\{
    \begin{array}{ccccccccc}
      2 x_1 &+& 2 x_2  &+&  x_3 &+& x_4 &=&7\\
      x_1 &-& x_2 &+& 2 x_3 &-& x_4 &=&1\\
      3 x_1 &+& 2 x_2 &-& 3 x_3 &-& 2 x_4&=& 4\\
      4 x_1 &+& 3 x_2  &+& 2 x_3 &+& x_4 &=&12\\
    \end{array}
\right.
\end{displaymath}

{\bf Solução:}

\begin{displaymath}
  \begin{array}{ccc}
      x_1 &=& 1\\
      x_2 &=& 2\\
      x_3 &=& 1\\
      x_4 &=& 0\\
  \end{array}
\end{displaymath}


\subsection*{Estratégias de Pivoteamento}

Obs: Curiosidade, não será cobrado na P2. (Não há garantias quanto à P3).

Pivoteamento parcial: procurar o pivô com maior magnitude (maior
módulo) na coluna, e trocar as linhas de modo que este seja o pivô da
eliminação.

Pivoteamento total (ou completo): procurar o pivô com maior magnitude
(maior módulo) entre as linhas e colunas, e trocar as linhas ou
colunas de modo que este seja o pivô da eliminação.

\end{document}
