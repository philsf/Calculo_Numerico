\everymath{\displaystyle}
%\documentclass[pdftex,a4paper]{article}
\documentclass[a4paper]{article}
%%classes: article, report, book, proc, amsproc

%%%%%%%%%%%%%%%%%%%%%%%%
%% Misc
% para acertar os acentos
\usepackage[brazilian]{babel}
%\usepackage[portuguese]{babel}
% \usepackage[english]{babel}
% \usepackage[T1]{fontenc}
% \usepackage[latin1]{inputenc}
\usepackage[utf8]{inputenc}
\usepackage{indentfirst}
\usepackage{fullpage}
% \usepackage{graphicx} %See PDF section
\usepackage{multicol}
\setlength{\columnseprule}{0.5pt}
\setlength{\columnsep}{20pt}
%%%%%%%%%%%%%%%%%%%%%%%%
%%%%%%%%%%%%%%%%%%%%%%%%
%% PDF support

\usepackage[pdftex]{color,graphicx}
% %% Hyper-refs
\usepackage[pdftex]{hyperref} % for printing
% \usepackage[pdftex,bookmarks,colorlinks]{hyperref} % for screen

%% \newif\ifPDF
%% \ifx\pdfoutput\undefined\PDFfalse
%% \else\ifnum\pdfoutput > 0\PDFtrue
%%      \else\PDFfalse
%%      \fi
%% \fi

%% \ifPDF
%%   \usepackage[T1]{fontenc}
%%   \usepackage{aeguill}
%%   \usepackage[pdftex]{graphicx,color}
%%   \usepackage[pdftex]{hyperref}
%% \else
%%   \usepackage[T1]{fontenc}
%%   \usepackage[dvips]{graphicx}
%%   \usepackage[dvips]{hyperref}
%% \fi

%%%%%%%%%%%%%%%%%%%%%%%%


%%%%%%%%%%%%%%%%%%%%%%%%
%% Math
\usepackage{amsmath,amsfonts,amssymb}
% para usar R de Real do jeito que o povo gosta
\usepackage{amsfonts} % \mathbb
% para usar as letras frescas como L de Espaco das Transf Lineares
% \usepackage{mathrsfs} % \mathscr

% Oferecer seno e tangente em pt, com os comandos usuais.
\providecommand{\sin}{} \renewcommand{\sin}{\hspace{2pt}\mathrm{sen}}
\providecommand{\tan}{} \renewcommand{\tan}{\hspace{2pt}\mathrm{tg}}

% dt of integrals = \ud t
\newcommand{\ud}{\mathrm{\ d}}
%%%%%%%%%%%%%%%%%%%%%%%%

\begin{document}

%%%%%%%%%%%%%%%%%%%%%%%%
%% Título e cabeçalho
%\noindent\parbox[c]{.15\textwidth}{\includegraphics[width=.15\textwidth]{logo}}\hfill
\parbox[c]{.825\textwidth}{\raggedright%
  \sffamily {\LARGE

Cálculo Numérico: Notas de aula: Integração Numérica

\par\bigskip}
{Prof: Felipe Figueiredo\par}
{\url{http://sites.google.com/site/proffelipefigueiredo}\par}
}

Versão: \verb|20150527|

%%%%%%%%%%%%%%%%%%%%%%%%


%%%%%%%%%%%%%%%%%%%%%%%%
\section*{Integração Numérica}

\subsection*{Pré-requisitos da aula}

Área do trapézio (lados $l_1,l_2$ e base $h$):
\begin{displaymath}
  A = \frac{h}{2}(l_1+l_2)
\end{displaymath}

Altura $y$ no gráfico de uma função (como localizar as coordenadas de
um ponto $(x,f(x))$ no gráfico de uma função (onde $y=f(x)$).

\subsection*{Método dos Trapézios}

\begin{enumerate}
\item Figura com gráfico, hachurar a área da integral
\item Figura com gráfico, hachurar a área do trapézio sobrescrito
\end{enumerate}

Com 1 trapézio:
\subsubsection*{Exemplo 1}

$\int_0^1 x^3 \ud x$ com 1 trapézio
\begin{displaymath}
  h = 1-0=1
\end{displaymath}

\begin{displaymath}
  A \approx \frac{h}{2}(f(x_0) + f(x_1)) = \frac{1}{2}(0^3 + 1^3) =
  \frac{1}{2} = 0.5
\end{displaymath}

Resposta: 0.5.

Obs: a resposta {\em exata} é 0.25!

\subsubsection*{Exemplo 2}

$\int_0^1 x^3 \ud x$ com 2 subdivisões

\begin{displaymath}
  h = \frac{1-0}{2} = 0.5
\end{displaymath}

\begin{displaymath}
  x_0 = 0, x_1=0.5, x_2=1
\end{displaymath}

\begin{displaymath}
  A \approx \frac{h}{2}(f(x_0) + f(x_1)) + \frac{h}{2}(f(x_1) + f(x_2))
\end{displaymath}
\begin{displaymath}
  = \frac{0.5}{2}(0^3 + 0.5^3 + 0.5^3 + 1^3) = 0.3125
\end{displaymath}

Resposta: 0.3125 é mais próxima de 0.25!

\subsubsection*{Exemplo 3}
$\int_0^1 x^3 \ud x$ com 4 subdivisões
\begin{displaymath}
  h = \frac{1-0}{4} = 0.25
\end{displaymath}

\begin{displaymath}
  x_0 = 0, x_1=0.25, x_2=0.5, x_3=0.75, x_4=1
\end{displaymath}

\begin{displaymath}
  A \approx \frac{h}{2} ((f(x_0) + f(x_1)) + (f(x_1) +
  f(x_2)) + (f(x_2) + f(x_3)) + (f(x_3) + f(x_4)) )
\end{displaymath}

\begin{displaymath}
  = \frac{0.25}{2}(0^3 + 0.25^3 + 0.25^3 + 0.5^3 + 0.5^3 + 0.75^3 +
  0.75^3 + 1^3)
\end{displaymath}

\begin{displaymath}
  = \frac{0.25}{2}(0^3 + 2\times 0.25^3 + 2\times 0.5^3  +2\times 0.75^3 + 1^3) = 0.265625
\end{displaymath}

Obs: Resposta 0.265625 bem mais próxima ainda de 0.25!

\subsubsection*{Exercício}

$\int_1^2 e^x \ud x$, com 4 subdivisões

R: 6.495075917

\begin{displaymath}
  h = \frac{2-1}{4} = 0.25
\end{displaymath}

\begin{displaymath}
  x_0 = 1, x_1=1.25, x_2=1.5, x_3=1.75, x_4=2
\end{displaymath}

\end{document}
