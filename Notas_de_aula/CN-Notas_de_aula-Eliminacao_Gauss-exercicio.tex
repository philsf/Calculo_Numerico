\everymath{\displaystyle}
%\documentclass[pdftex,a4paper]{article}
\documentclass[a4paper]{article}
%%classes: article, report, book, proc, amsproc

%%%%%%%%%%%%%%%%%%%%%%%%
%% Misc
% para acertar os acentos
\usepackage[brazilian]{babel} 
%\usepackage[portuguese]{babel} 
% \usepackage[english]{babel}
% \usepackage[T1]{fontenc}
% \usepackage[latin1]{inputenc}
\usepackage[utf8]{inputenc}
\usepackage{indentfirst}
\usepackage{fullpage}
% \usepackage{graphicx} %See PDF section
\usepackage{multicol}
\setlength{\columnseprule}{0.5pt}
\setlength{\columnsep}{20pt}
%%%%%%%%%%%%%%%%%%%%%%%%
%%%%%%%%%%%%%%%%%%%%%%%%
%% PDF support

\usepackage[pdftex]{color,graphicx}
% %% Hyper-refs
\usepackage[pdftex]{hyperref} % for printing
% \usepackage[pdftex,bookmarks,colorlinks]{hyperref} % for screen

%% \newif\ifPDF
%% \ifx\pdfoutput\undefined\PDFfalse
%% \else\ifnum\pdfoutput > 0\PDFtrue
%%      \else\PDFfalse
%%      \fi
%% \fi

%% \ifPDF
%%   \usepackage[T1]{fontenc}
%%   \usepackage{aeguill}
%%   \usepackage[pdftex]{graphicx,color}
%%   \usepackage[pdftex]{hyperref}
%% \else
%%   \usepackage[T1]{fontenc}
%%   \usepackage[dvips]{graphicx}
%%   \usepackage[dvips]{hyperref}
%% \fi

%%%%%%%%%%%%%%%%%%%%%%%%


%%%%%%%%%%%%%%%%%%%%%%%%
%% Math
\usepackage{amsmath,amsfonts,amssymb}
% para usar R de Real do jeito que o povo gosta
\usepackage{amsfonts} % \mathbb
% para usar as letras frescas como L de Espaco das Transf Lineares
% \usepackage{mathrsfs} % \mathscr

% Oferecer seno e tangente em pt, com os comandos usuais.
\providecommand{\sin}{} \renewcommand{\sin}{\hspace{2pt}\mathrm{sen}}
\providecommand{\tan}{} \renewcommand{\tan}{\hspace{2pt}\mathrm{tg}}

% dt of integrals = \ud t
\newcommand{\ud}{\mathrm{\ d}}
%%%%%%%%%%%%%%%%%%%%%%%%



\begin{document}

%%%%%%%%%%%%%%%%%%%%%%%%
%% Título e cabeçalho
%\noindent\parbox[c]{.15\textwidth}{\includegraphics[width=.15\textwidth]{logo}}\hfill
\parbox[c]{.825\textwidth}{\raggedright%
  \sffamily {\LARGE

Cálculo Numérico: Notas de Aula: Exercício de Eliminação de Gauss

\par\bigskip}
{Prof: Felipe Figueiredo\par}
{\url{http://sites.google.com/site/proffelipefigueiredo}\par}
}

Versão: \verb|20150510|

%%%%%%%%%%%%%%%%%%%%%%%%


%%%%%%%%%%%%%%%%%%%%%%%%

\section{Exercício}

Resolva o sistema linear utilizando o método de Eliminação de Gauss:

\begin{displaymath}
  \left\{
    \begin{array}{ccccccccc}
      2 x_1 &+& 2 x_2  &+&  x_3 &+& x_4 &=&7\\
      x_1 &-& x_2 &+& 2 x_3 &-& x_4 &=&1\\
      3 x_1 &+& 2 x_2 &-& 3 x_3 &-& 2 x_4&=& 4\\
      4 x_1 &+& 3 x_2  &+& 2 x_3 &+& x_4 &=&12\\
    \end{array}
\right.
\end{displaymath}

{\bf Solução:}

\begin{displaymath}
  \begin{array}{ccc}
      x_1 &=& 1\\
      x_2 &=& 2\\
      x_3 &=& 1\\
      x_4 &=& 0\\
  \end{array}
\end{displaymath}

\subsection{Frações}

Caso o aluno opte por resolver usando frações:

\subsubsection{Matriz aumentada}

\begin{displaymath}
  \begin{bmatrix}
    2 & 2 & 1 & 1 & | & 7\\
    1 & -1 & 2 & -1 & | & 1\\
    3 & 2 & -3 & -2 & | & 4\\
    4 & 3 & 2 & 1 & | & 12\\
  \end{bmatrix}
\end{displaymath}

\subsubsection{Etapa 1}

\begin{itemize}
\item pivô: $a_{11} = 2$

\item multiplicadores:

$m_{21} = \frac{1}{2}$

$m_{31} = \frac{3}{2}$

$m_{41} = 2$

\end{itemize}
\begin{displaymath}
  \begin{bmatrix}
    2 & 2 & 1 & 1 & | & 7\\
    0 & -2 & \frac{3}{2} & \frac{-3}{2} & | & \frac{-5}{2}\\
    0 & -1 & \frac{-9}{2} & \frac{-7}{2} & | & \frac{-13}{2}\\
    0 & -1 & 0 & -1 & | & -2\\
  \end{bmatrix}
\end{displaymath}

\subsubsection{Etapa 2}

\begin{itemize}
\item pivô: $a_{22} = -2$

\item multiplicadores:

$m_{32}=\frac{-1}{-2} = \frac{1}{2}$

$m_{42} = \frac{-1}{-2} = \frac{1}{2}$

\end{itemize}
\begin{displaymath}
  \begin{bmatrix}
    2 & 2 & 1 & 1 & | & 7\\
    0 & -2 & \frac{3}{2} & \frac{-3}{2} & | & \frac{-5}{2}\\
    0 & 0 & \frac{-21}{4} & \frac{-11}{4} & | & \frac{-21}{4}\\
    0 & 0 & \frac{-3}{4} & \frac{-1}{4} & | & \frac{-3}{4}\\
  \end{bmatrix}
\end{displaymath}

\subsubsection{Etapa 3}

\begin{itemize}
\item pivô: $a_{33} = \frac{-21}{4}$

\item multiplicador: $m_{43}=\frac{\frac{-3}{4}}{\frac{-21}{4}} =
\frac{-3}{-21} = \frac{1}{7}$

\end{itemize}
\begin{displaymath}
  \begin{bmatrix}
    2 & 2 & 1 & 1 & | & 7\\
    0 & -2 & \frac{3}{2} & \frac{-3}{2} & | & \frac{-5}{2}\\
    0 & 0 & \frac{-21}{4} & \frac{-11}{4} & | & \frac{-21}{4}\\
    0 & 0 & 0 & \frac{1}{7} & | & 0\\
  \end{bmatrix}
\end{displaymath}

\subsubsection{Resolução}

Sistema escalonado equivalente ao original:

\begin{displaymath}
  \left\{
    \begin{array}{ccccccccc}
      2 x_1 &+& 2 x_2  &+&  x_3 &+& x_4 &=&7\\
      & -& 2 x_2 &+& \frac{3}{2} x_3 &-& \frac{3}{2} x_4 &=&\frac{-5}{2}\\
      & & &-& \frac{21}{4} x_3 &-& \frac{11}{4} x_4&=& \frac{-21}{4}\\
      & & & &  & & \frac{1}{7}x_4 &=&0\\
    \end{array}
\right.
\end{displaymath}

Aplicando o método de ``substituição para trás'', encontramos:

\begin{displaymath}
  \frac{1}{7} x_4 = 0 \Rightarrow x_4 = 0
\end{displaymath}

\begin{displaymath}
  \frac{-21}{4} x_3 + (0) = \frac{-21}{4} \Rightarrow  x_3 = 1
\end{displaymath}

\begin{displaymath}
  -2 x_2 + \frac{3}{2} (1) + (0) = \frac{-5}{2} \Rightarrow x_2 = 2
\end{displaymath}

\begin{displaymath}
  2 x_1 + 2 (2) + (1) + (0) = 7 \Rightarrow x_1 = 1
\end{displaymath}


\newpage
\subsection{Calculadora}

Caso o aluno opte por resolver usando a calculadora, consultar os
pivôs e multiplicadores acima, e conferir as matrizes em cada etapa
conforme abaixo.

Obs: Os cálculos a seguir foram feitos usando o software livre Octave,
semelhante ao Matlab.

\subsubsection{Matriz aumentada}

\begin{verbatim}
A0 =
    2    2    1    1    7
    1   -1    2   -1    1
    3    2   -3   -2    4
    4    3    2    1   12
\end{verbatim}

\subsubsection{Etapa 1}

\begin{verbatim}
A1 =
   2.00000   2.00000   1.00000   1.00000   7.00000
   0.00000  -2.00000   1.50000  -1.50000  -2.50000
   0.00000  -1.00000  -4.50000  -3.50000  -6.50000
   0.00000  -1.00000   0.00000  -1.00000  -2.00000
\end{verbatim}

\subsubsection{Etapa 2}

\begin{verbatim}
A2 =
   2.00000   2.00000   1.00000   1.00000   7.00000
   0.00000  -2.00000   1.50000  -1.50000  -2.50000
   0.00000   0.00000  -5.25000  -2.75000  -5.25000
   0.00000   0.00000  -0.75000  -0.25000  -0.75000
\end{verbatim}

\subsubsection{Etapa 3}

\begin{verbatim}
A3 =

   2.00000   2.00000   1.00000   1.00000   7.00000
   0.00000  -2.00000   1.50000  -1.50000  -2.50000
   0.00000   0.00000  -5.25000  -2.75000  -5.25000
   0.00000   0.00000   0.00000   0.14286   0.00000
\end{verbatim}

\end{document}
