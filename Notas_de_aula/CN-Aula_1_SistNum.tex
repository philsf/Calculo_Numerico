\everymath{\displaystyle}
%\documentclass[pdftex,a4paper]{article}
\documentclass[a4paper]{article}
%%classes: article, report, book, proc, amsproc

%%%%%%%%%%%%%%%%%%%%%%%%
%% Misc
% para acertar os acentos
\usepackage[brazilian]{babel} 
%\usepackage[portuguese]{babel} 
% \usepackage[english]{babel}
% \usepackage[T1]{fontenc}
% \usepackage[latin1]{inputenc}
\usepackage[utf8]{inputenc}
\usepackage{indentfirst}
\usepackage{fullpage}
% \usepackage{graphicx} %See PDF section
\usepackage{multicol}
\setlength{\columnseprule}{0.5pt}
\setlength{\columnsep}{20pt}
%%%%%%%%%%%%%%%%%%%%%%%%
%%%%%%%%%%%%%%%%%%%%%%%%
%% PDF support

\usepackage[pdftex]{color,graphicx}
% %% Hyper-refs
\usepackage[pdftex]{hyperref} % for printing
% \usepackage[pdftex,bookmarks,colorlinks]{hyperref} % for screen

%% \newif\ifPDF
%% \ifx\pdfoutput\undefined\PDFfalse
%% \else\ifnum\pdfoutput > 0\PDFtrue
%%      \else\PDFfalse
%%      \fi
%% \fi

%% \ifPDF
%%   \usepackage[T1]{fontenc}
%%   \usepackage{aeguill}
%%   \usepackage[pdftex]{graphicx,color}
%%   \usepackage[pdftex]{hyperref}
%% \else
%%   \usepackage[T1]{fontenc}
%%   \usepackage[dvips]{graphicx}
%%   \usepackage[dvips]{hyperref}
%% \fi

%%%%%%%%%%%%%%%%%%%%%%%%


%%%%%%%%%%%%%%%%%%%%%%%%
%% Math
\usepackage{amsmath,amsfonts,amssymb}
% para usar R de Real do jeito que o povo gosta
\usepackage{amsfonts} % \mathbb
% para usar as letras frescas como L de Espaco das Transf Lineares
% \usepackage{mathrsfs} % \mathscr

% Oferecer seno e tangente em pt, com os comandos usuais.
\providecommand{\sin}{} \renewcommand{\sin}{\hspace{2pt}\mathrm{sen}}
\providecommand{\tan}{} \renewcommand{\tan}{\hspace{2pt}\mathrm{tg}}

% dt of integrals = \ud t
\newcommand{\ud}{\mathrm{\ d}}
%%%%%%%%%%%%%%%%%%%%%%%%



\begin{document}

%%%%%%%%%%%%%%%%%%%%%%%%
%% Título e cabeçalho
%\noindent\parbox[c]{.15\textwidth}{\includegraphics[width=.15\textwidth]{logo}}\hfill
\parbox[c]{.825\textwidth}{\raggedright%
  \sffamily {\LARGE

Cálculo Numérico: Notas de Aula: Sistemas Binário, decimal e hexadecimal

\par\bigskip}
{Prof: Felipe Figueiredo\par}
{\url{http://sites.google.com/site/proffelipefigueiredo}}

\vspace{1cm}
}
%%%%%%%%%%%%%%%%%%%%%%%%


%%%%%%%%%%%%%%%%%%%%%%%%

\section*{1.2.1 Conversão de números nos sistemas decimal e binário}

\subsection*{Números inteiros}
\subsubsection*{Decomposição em potências da base}
 
Decimal:

$35 = 30+5 = 3\times 10^1 + 5\times 10^0$

$2015 = 2000 + 0+ 10 + 5 = 2\times 10^3+0\times 10^2+1\times 10^1+5\times 10^0$

Ex: $347 = 3\times 10^2 + 4\times 10^1+7\times 10^0$

Binária:

$10 = 1\times 2^1+0\times 2^0$

$101=1\times 2^2+0\times 2^1+1\times 2^0$

Ex: $10111 = 1\times 2^4  + 0\times 2^3 + 1\times 2^2 + 1\times 2^1 + 1\times 2^0$

\subsubsection*{Conversão para a base decimal}

$(10)_2 = 1\times 2^1+0\times 2^0 = 2+0 = (2)_{10}$

$(101)_2 = 4+0+1 = 5$

Ex: $(10111)_2 = 16 + 0 + 4 + 2 +1 = 23$

\subsubsection*{Conversão para a base binária}

$(10)_{10} = 1010_2$

$(13)_{10} = 1101_2$

$(17)_{10} = 10001_2$

Ex: $(23)_{10} = 10111_2$

\subsection*{Números fracionários}
\subsubsection*{Decimal para binário}

$(0.5)_{10} = 0,1$

$(0.25)_{10} = 0,01$

$(0.125)_{10} = 0,001$

$(0.375)_{10} = 0,011$

Ex: $(0.625)_{10} = 0,101$
 
Falar de dízima periódica 

$(0.1)_{10} = 0.000110011001100110011\ldots$ (dízima em 0011) 
 
Ex: $(0.11)_{10} = 0.0001110000\ldots$
 
Curiosidade (relacionado à ATPS) Binário para decimal: 

$(0.000111)_2 = 0.109375$

\section*{O sistema hexadecimal}

\subsubsection*{Inteiros em base Hexadecimal}

\begin{tabular}{ccc}
  0 &=& 0\\
  ...\\
  9 &=& 9\\
  A &=& 10\\
  B &=& 11\\
  C &=& 12\\
  D &=& 13\\
  E &=& 14\\
  F &=& 15\\
\end{tabular}

\subsubsection*{Decomposição em potências da base}

$1A = 1\times 16^1 + A\times 16^0  = 1\times 16^1 + 10\times 16^0$

$7B3 = 7\times 16^2 + B\times 16^1 + 3\times 16^0  = 6\times 16^2 + 11\times 16^1 + 3$

Ex: $CB = C\times 16^1 + B\times 16^0$

Ex: $BEBE = B\times 16^3 + E\times 16^2 + B\times 16^1 + E\times 16^0$

\subsubsection*{Conversão de decimal para hex}

$(16)_{10} = 10$

$(26)_{10} = 1A$

$(27)_{10} = 1B$

\subsubsection*{Conversão de hex para decimal}

$1A = 1\times 16^1 + A\times 16^0  = 1\times 16^1 + 10\times 16^0  = 16 + 10 = 26$

$7B3 = 7\times 16^2 + 11\times 16^1 + 3\times 16^0  = 6\times 16^2 + 11\times 16^1 + 3 = 1792 + 176 + 3 =
1971$

$CB = 12\times 16^1 + 11\times 16^0  = 192 + 11 = 203$

$BEBE = B\times 16^3 + E\times 16^2 + B\times 16^1 + E\times 16^0  = 11\times 4096 + 14\times 256 + 11\times 16 + 14
 = 45056 + 3584 + 176 + 14 = 48830$
 
$BABACA = 12237514$

\end{document}
