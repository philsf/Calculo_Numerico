\everymath{\displaystyle}
%\documentclass[pdftex,a4paper]{article}
\documentclass[a4paper]{article}
%%classes: article, report, book, proc, amsproc

%%%%%%%%%%%%%%%%%%%%%%%%
%% Misc
% para acertar os acentos
\usepackage[brazilian]{babel} 
%\usepackage[portuguese]{babel} 
% \usepackage[english]{babel}
% \usepackage[T1]{fontenc}
% \usepackage[latin1]{inputenc}
\usepackage[utf8]{inputenc}
\usepackage{indentfirst}
\usepackage{fullpage}
% \usepackage{graphicx} %See PDF section
\usepackage{multicol}
\setlength{\columnseprule}{0.5pt}
\setlength{\columnsep}{20pt}
%%%%%%%%%%%%%%%%%%%%%%%%
%%%%%%%%%%%%%%%%%%%%%%%%
%% PDF support

\usepackage[pdftex]{color,graphicx}
% %% Hyper-refs
\usepackage[pdftex]{hyperref} % for printing
% \usepackage[pdftex,bookmarks,colorlinks]{hyperref} % for screen

%% \newif\ifPDF
%% \ifx\pdfoutput\undefined\PDFfalse
%% \else\ifnum\pdfoutput > 0\PDFtrue
%%      \else\PDFfalse
%%      \fi
%% \fi

%% \ifPDF
%%   \usepackage[T1]{fontenc}
%%   \usepackage{aeguill}
%%   \usepackage[pdftex]{graphicx,color}
%%   \usepackage[pdftex]{hyperref}
%% \else
%%   \usepackage[T1]{fontenc}
%%   \usepackage[dvips]{graphicx}
%%   \usepackage[dvips]{hyperref}
%% \fi

%%%%%%%%%%%%%%%%%%%%%%%%


%%%%%%%%%%%%%%%%%%%%%%%%
%% Math
\usepackage{amsmath,amsfonts,amssymb}
% para usar R de Real do jeito que o povo gosta
\usepackage{amsfonts} % \mathbb
% para usar as letras frescas como L de Espaco das Transf Lineares
% \usepackage{mathrsfs} % \mathscr

% Oferecer seno e tangente em pt, com os comandos usuais.
\providecommand{\sin}{} \renewcommand{\sin}{\hspace{2pt}\mathrm{sen}}
\providecommand{\tan}{} \renewcommand{\tan}{\hspace{2pt}\mathrm{tg}}

% dt of integrals = \ud t
\newcommand{\ud}{\mathrm{\ d}}
%%%%%%%%%%%%%%%%%%%%%%%%



\begin{document}

%%%%%%%%%%%%%%%%%%%%%%%%
%% Título e cabeçalho
%\noindent\parbox[c]{.15\textwidth}{\includegraphics[width=.15\textwidth]{logo}}\hfill
\parbox[c]{.825\textwidth}{\raggedright%
  \sffamily {\LARGE

Cálculo Numérico: Notas de Aula: Fatoração LU

\par\bigskip}
{Prof: Felipe Figueiredo\par}
{\url{http://sites.google.com/site/proffelipefigueiredo}\par}
}

Versão: \verb|20150511|

%%%%%%%%%%%%%%%%%%%%%%%%


%%%%%%%%%%%%%%%%%%%%%%%%

\section*{3.2.3 Fatoração LU}

\subsection*{Cálculo dos Fatores L e U}

A matriz A antes de iniciar a eliminação de Gauss (``etapa 0''):

\begin{displaymath}
  A^{(0)} = \begin{bmatrix}
    a_{11} & a_{12} & a_{13} \\
    a_{21} & a_{22} & a_{23} \\
    a_{31} & a_{32} & a_{33} \\
  \end{bmatrix}
\end{displaymath}

Na etapa 1 são feitas as operações:

$L_2 = L_2 - m_{21}L_1$

$L_3 = L_3 - m_{31}L_1$

que resultam na matriz:

\begin{displaymath}
  A^{(1)} = \begin{bmatrix}
    a_{11} & a_{12} & a_{13} \\
    0 & a_{22} & a_{23} \\
    0 & a_{32} & a_{33} \\
  \end{bmatrix}
\end{displaymath}

Na etapa 2 é feita a operação

$L_3 = L_3 - m_{32}L_1$

Ao final da etapa 2, temos uma matriz triangular superior:

\begin{displaymath}
  A^{(2)} = \begin{bmatrix}
    a_{11} & a_{12} & a_{13} \\
    0 & a_{22} & a_{23} \\
    0 & 0 & a_{33} \\
  \end{bmatrix}
\end{displaymath}

Partindo da matriz $A^{(0)}$ até chegar na matriz $A^{(2)}$ utilizamos
os multiplicadores $m_{21}$, $m_{31}$ e $m_{32}$.

Podemos incorporar todos esses multiplicadores nas suas respectivas
posições de uma matriz identidade, e obter uma matriz $L$ que é
triangular inferior com 1 na diagonal:

\begin{displaymath}
  L = \begin{bmatrix}
    1 & 0 & 0 \\
    m_{21} & 1 & 0 \\
    m_{31} & m_{32} & 1 \\
  \end{bmatrix}
\end{displaymath}

E a matriz $U$ que é triangular superior é a etapa final da eliminação:

\begin{displaymath}
  U = A^{(2)} = \begin{bmatrix}
    a_{11} & a_{12} & a_{13} \\
    0 & a_{22} & a_{23} \\
    0 & 0 & a_{33} \\
  \end{bmatrix}
\end{displaymath}

de modo que a matriz original $A^{(0)}$ é o produto

\begin{displaymath}
  A = LU
\end{displaymath}

\subsection*{Resolução do Sistema Linear Ax=b Usando a Fatoração LU de
A}

\begin{displaymath}
  Ax = b \iff (LU)x=b
\end{displaymath}

Seja $y = Ux$. 

\begin{displaymath}
  Ax = b \iff (LU)x=b \iff L(Ux)=b \iff L(y) = b
\end{displaymath}

Resolvendo da esquerda para a direita, temos:

\begin{enumerate}
\item $Ly = b$
\item $Ux = y$
\end{enumerate}

\subsection*{Exemplo}

\begin{displaymath}
  \left\{
  \begin{array}{ccccc}
    3x_1 &+ 2x_2 &+ 4x_3 &=&1\\
    x_1 &+ x_2 &+ 2x_3 &=& 2\\
    4x_1 &+ 3x_2 &+ 2x_3 &=& 3\\
  \end{array}
\right.
\end{displaymath}

\begin{displaymath}
  A = A^{(0)} = \begin{bmatrix}
    3 & 2 & 4 \\
    1 & 1 & 2 \\
    4 & 3 & 2 \\
  \end{bmatrix}
\end{displaymath}


\subsubsection*{Etapa 1}
pivô : $3$

multiplicadores 

$m_{21}= \frac{1}{3}$

$m_{31} = \frac{4}{3}$

E transformamos $A^{(0)}$ na seguinte matriz $A^{(1)}$:

\begin{displaymath}
  A^{(1)} = \begin{bmatrix}
    3 & 2 & 4\\
    0 & \frac{1}{3} & \frac{2}{3}\\
    0 & \frac{1}{3} & \frac{-10}{3}\\
  \end{bmatrix}
\end{displaymath}

\subsubsection*{Etapa 2}

pivô: $\frac{1}{3}$

multiplicador: $\frac{\frac{1}{3}}{\frac{1}{3}} = 1$

E transformamos a matriz $A^{(1)}$ na matriz $A^{(2)}$:

\begin{displaymath}
  A^{(1)} = \begin{bmatrix}
    3 & 2 & 4\\
    0 & \frac{1}{3} & \frac{2}{3}\\
    0 & 0 & -4\\
  \end{bmatrix}
\end{displaymath}

Os fatores $L$ e $U$ são:

\begin{displaymath}
  L = \begin{bmatrix}
    1 & 0 & 0\\
    \frac{1}{3} & 1 & 0\\
    \frac{4}{3} & 1 & 1\\
  \end{bmatrix}
\end{displaymath}

\begin{displaymath}
  U = \begin{bmatrix}
    3 & 2 & 4\\
    0 & \frac{1}{3} & \frac{2}{3}\\
    0 & 0 & -4\\
  \end{bmatrix}
\end{displaymath}


\subsubsection*{Resolução}

Resolvendo $L(Ux)=b$

\begin{enumerate}
\item $Ly=b$

  \begin{displaymath}
    \left\{
      \begin{array}{ccccc}
        y_1 & & &=&1\\
        \frac{1}{3} y_1 &+ y_2 &  &=& 2\\
        \frac{4}{3} y_1 &+ y_2 & x_3 &=& 3\\
      \end{array}
    \right.
  \end{displaymath}
  
Solução: $y_1 = 1, y_2 = \frac{5}{3}, y_3 = 0$

\item $Ux = y$

\begin{displaymath}
  \left\{
  \begin{array}{ccccc}
    3x_1 &+ 2x_2 &+ 4x_3 &=&1\\
     & \frac{1}{3} x_2 &+ \frac{2}{3} x_3 &=& \frac{5}{3}\\
     & & -4x_3 &=& 0\\
  \end{array}
\right.
\end{displaymath}

Solução: $x_1=-3, x_2 = 5, x_3 = 0$
\end{enumerate}

\subsection*{Justificativa do método}

As operações nas linhas da etapa 1 podem ser incorporadas em uma matriz $M^{(0)}$:

\begin{displaymath}
  M^{(0)} = \begin{bmatrix}
    1 & 0 & 0 \\
    -m_{21} & 1 & 0 \\
    -m_{31} & 0 & 1 \\
  \end{bmatrix}
\end{displaymath}

Ou seja, a matriz $M^{(0)}$ é a identidade $I$ com os multiplicadores
da primeira etapa na primeira coluna em suas posições respectivas.

% \begin{displaymath}
%   I = \begin{bmatrix}
%     1 & 0 & 0 \\
%     0 & 1 & 0 \\
%     0 & 0 & 1 \\
%   \end{bmatrix}
% \end{displaymath}

Assim, 

\begin{displaymath}
   M^{(0)}A^{(0)} = A^{(1)}
\end{displaymath}
ou:
\begin{displaymath}
  \begin{bmatrix}
    1 & 0 & 0 \\
    -m_{21} & 1 & 0 \\
    -m_{31} & 0 & 1 \\
  \end{bmatrix}
  \begin{bmatrix}
    a_{11} & a_{12} & a_{13} \\
    a_{21} & a_{22} & a_{23} \\
    a_{31} & a_{32} & a_{33} \\
  \end{bmatrix} = \begin{bmatrix}
    a_{11} & a_{12} & a_{13} \\
    0 & a_{22} & a_{23} \\
    0 & a_{32} & a_{33} \\
  \end{bmatrix}
\end{displaymath}

A segunda etapa pode ser incorporada analogamente:

\begin{displaymath}
   M^{(1)}A^{(1)} = A^{(2)}
\end{displaymath}
ou:
\begin{displaymath}
  \begin{bmatrix}
    1 & 0 & 0 \\
    0 & 1 & 0 \\
    0 & -m_{32} & 1 \\
  \end{bmatrix}
  \begin{bmatrix}
    a_{11} & a_{12} & a_{13} \\
    0 & a_{22} & a_{23} \\
    0 & a_{32} & a_{33} \\
  \end{bmatrix} = \begin{bmatrix}
    a_{11} & a_{12} & a_{13} \\
    0 & a_{22} & a_{23} \\
    0 & 0 & a_{33} \\
  \end{bmatrix}
\end{displaymath}

Concluímos que o método de Eliminação de Gauss é equivalente a:

\begin{displaymath}
  A = A^{(0)}
\end{displaymath}

\begin{displaymath}
  A^{(1)} = M^{(0)}A^{(0)} = M^{(0)}A
\end{displaymath}

\begin{displaymath}
  A^{(2)} = M^{(1)}A^{(1)} = M^{(1)}M^{(0)}A
\end{displaymath}

Assim, a matriz $L$, triangular inferior com 1 na diagonal é:

\begin{displaymath}
  L = \begin{bmatrix}
    1 & 0 & 0 \\
    m_{21} & 1 & 0 \\
    m_{31} & m_{32} & 1 \\
  \end{bmatrix}
\end{displaymath}

E a matriz $U$ é a etapa final:

\begin{displaymath}
  U = A^{(2)} = \begin{bmatrix}
    a_{11} & a_{12} & a_{13} \\
    0 & a_{22} & a_{23} \\
    0 & 0 & a_{33} \\
  \end{bmatrix}
\end{displaymath}

\end{document}
