\everymath{\displaystyle}
%\documentclass[pdftex,a4paper]{article}
\documentclass[a4paper]{article}
%%classes: article, report, book, proc, amsproc

%%%%%%%%%%%%%%%%%%%%%%%%
%% Misc
% para acertar os acentos
\usepackage[brazilian]{babel} 
%\usepackage[portuguese]{babel} 
% \usepackage[english]{babel}
% \usepackage[T1]{fontenc}
% \usepackage[latin1]{inputenc}
\usepackage[utf8]{inputenc}
\usepackage{indentfirst}
\usepackage{fullpage}
% \usepackage{graphicx} %See PDF section
\usepackage{multicol}
\setlength{\columnseprule}{0.5pt}
\setlength{\columnsep}{20pt}
%%%%%%%%%%%%%%%%%%%%%%%%
%%%%%%%%%%%%%%%%%%%%%%%%
%% PDF support

\usepackage[pdftex]{color,graphicx}
% %% Hyper-refs
\usepackage[pdftex]{hyperref} % for printing
% \usepackage[pdftex,bookmarks,colorlinks]{hyperref} % for screen

%% \newif\ifPDF
%% \ifx\pdfoutput\undefined\PDFfalse
%% \else\ifnum\pdfoutput > 0\PDFtrue
%%      \else\PDFfalse
%%      \fi
%% \fi

%% \ifPDF
%%   \usepackage[T1]{fontenc}
%%   \usepackage{aeguill}
%%   \usepackage[pdftex]{graphicx,color}
%%   \usepackage[pdftex]{hyperref}
%% \else
%%   \usepackage[T1]{fontenc}
%%   \usepackage[dvips]{graphicx}
%%   \usepackage[dvips]{hyperref}
%% \fi

%%%%%%%%%%%%%%%%%%%%%%%%


%%%%%%%%%%%%%%%%%%%%%%%%
%% Math
\usepackage{amsmath,amsfonts,amssymb}
% para usar R de Real do jeito que o povo gosta
\usepackage{amsfonts} % \mathbb
% para usar as letras frescas como L de Espaco das Transf Lineares
% \usepackage{mathrsfs} % \mathscr

% Oferecer seno e tangente em pt, com os comandos usuais.
\providecommand{\sin}{} \renewcommand{\sin}{\hspace{2pt}\mathrm{sen}}
\providecommand{\tan}{} \renewcommand{\tan}{\hspace{2pt}\mathrm{tg}}

% dt of integrals = \ud t
\newcommand{\ud}{\mathrm{\ d}}
%%%%%%%%%%%%%%%%%%%%%%%%



\begin{document}

%%%%%%%%%%%%%%%%%%%%%%%%
%% Título e cabeçalho
%\noindent\parbox[c]{.15\textwidth}{\includegraphics[width=.15\textwidth]{logo}}\hfill
\parbox[c]{.825\textwidth}{\raggedright%
  \sffamily {\LARGE

Cálculo Numérico: Notas de Aula: Ponto Flutuante e Erros

\par\bigskip}
{Prof: Felipe Figueiredo\par}
{\url{http://sites.google.com/site/proffelipefigueiredo}\par}
}

Versão: \verb|20150404|

%%%%%%%%%%%%%%%%%%%%%%%%


%%%%%%%%%%%%%%%%%%%%%%%%

\section*{1.2.2 Aritmética de Ponto Flutuante}
IEEE – Institute of Electrical and Electronic Engineers

\subsection*{Representação em ponto flutuante normalizada}
$r = sinal \times mantissa \times base^{expoente} = \pm 0.ddddd...dd \times \beta ^ e$

Forma normalizada: um zero antes da vírgula, primeiro dígito depois da vírgula não-nulo.

$0.35 = 0.35\times 10^0$

$5.47 = 0.547\times 10^1$

$-123.456 = -12.3456\times 10^1 = -1.23456\times 10^2 = -0.123456\times 10^3$ (normalizada)

Normalizar
Ex: $2013.54 = 0.201354\times 10^4$

Ex: $0.000502 = 0.502\times 10^{-3}$

\subsection*{Sistema de ponto flutuante}
$f(\beta, t, m, M)$ , base $\beta$, t dígitos significativos, $-m \le e \le M$

Considere uma máquina com o sistema $f(10,3,5,5)$
 
\begin{itemize}
\item Qual é o menor número positivo?
 
$0.100\times 10^{-5} = 10^{-6}$

Representar um número menor que esse: {\bf underflow}
 
\item Qual é o maior número positivo?

$0.999\times 105  =99900$

Representar um número maior que esse: {\bf overflow}
\end{itemize}

\section*{1.3.1 Erros Absolutos e Relativos}

Erro absoluto: diferença entre o valor exato $x$ e o valor aproximado
$\bar{x}$

\begin{displaymath}
  EA = x - \bar{x}
\end{displaymath}
 
Cota superior para erro 

$ 3.14 \le \pi \le 3.15$

Então: 

$\pi \in (3.14,3.15)$

$|EA_\pi| = |\pi - \bar{\pi}| < 0.01$
 
Erros absolutos são sempre iguais? Como comparar essas cotas para erros absolutos?  
 
Seja $\bar{x} = 2112.9$ tal que $|EA|<0.1$. Então $\bar{x} \in
(2112.8,2113)$.

Seja $\bar{y} = 5.3$ tal que $|EA|<0.1$. Então $\bar{y} \in
(5.2,5.4)$.
 
\begin{displaymath}
  ER = \frac{EA}{\bar{x}} = \frac{x-\bar{x}}{\bar{x}}
\end{displaymath}

$|ER_x| < \frac{0.1}{2112.9} = 0.000047328 \approx 0.47\times 10^{-4}$

$|ER_y| < \frac{0.1}{5.3} = 0.018867925 \approx 0.19\times 10^1$
 
Portanto $x$ é representado com maior precisão que $y$.

\section*{1.3.2 Erros de Arredondamento e Truncamento}
E quanto o número não pode ser representado na máquina, por falta de precisão?
 
$f(10,4,5,5)$ e $x = 234.57$

Normalizando, temos $x=0.23457\times 10^3$ (5 dígitos na mantissa!)
 
Truncamento: $\bar{x}=0.2345\times 10^3$

Arredondamento $\bar{x}=0.2346\times 10^3$

$x = 0.2345\times 10^3 + 0.00007\times 10^3 = 0.2345\times 10^3 + 0.7\times 10^{-1}$



\url{http://docs.oracle.com/cd/E19957-01/806-3568/ncg_goldberg.html}

\end{document}
