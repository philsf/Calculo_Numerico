\everymath{\displaystyle}
%\documentclass[pdftex,a4paper]{article}
\documentclass[a4paper]{article}
%%classes: article, report, book, proc, amsproc

%%%%%%%%%%%%%%%%%%%%%%%%
%% Misc
% para acertar os acentos
\usepackage[brazilian]{babel} 
%\usepackage[portuguese]{babel} 
% \usepackage[english]{babel}
% \usepackage[T1]{fontenc}
% \usepackage[latin1]{inputenc}
\usepackage[utf8]{inputenc}
\usepackage{indentfirst}
\usepackage{fullpage}
% \usepackage{graphicx} %See PDF section
\usepackage{multicol}
\setlength{\columnseprule}{0.5pt}
\setlength{\columnsep}{20pt}
%%%%%%%%%%%%%%%%%%%%%%%%
%%%%%%%%%%%%%%%%%%%%%%%%
%% PDF support

\usepackage[pdftex]{color,graphicx}
% %% Hyper-refs
\usepackage[pdftex]{hyperref} % for printing
% \usepackage[pdftex,bookmarks,colorlinks]{hyperref} % for screen

%% \newif\ifPDF
%% \ifx\pdfoutput\undefined\PDFfalse
%% \else\ifnum\pdfoutput > 0\PDFtrue
%%      \else\PDFfalse
%%      \fi
%% \fi

%% \ifPDF
%%   \usepackage[T1]{fontenc}
%%   \usepackage{aeguill}
%%   \usepackage[pdftex]{graphicx,color}
%%   \usepackage[pdftex]{hyperref}
%% \else
%%   \usepackage[T1]{fontenc}
%%   \usepackage[dvips]{graphicx}
%%   \usepackage[dvips]{hyperref}
%% \fi

%%%%%%%%%%%%%%%%%%%%%%%%


%%%%%%%%%%%%%%%%%%%%%%%%
%% Math
\usepackage{amsmath,amsfonts,amssymb}
% para usar R de Real do jeito que o povo gosta
\usepackage{amsfonts} % \mathbb
% para usar as letras frescas como L de Espaco das Transf Lineares
% \usepackage{mathrsfs} % \mathscr

% Oferecer seno e tangente em pt, com os comandos usuais.
\providecommand{\sin}{} \renewcommand{\sin}{\hspace{2pt}\mathrm{sen}}
\providecommand{\tan}{} \renewcommand{\tan}{\hspace{2pt}\mathrm{tg}}

% dt of integrals = \ud t
\newcommand{\ud}{\mathrm{\ d}}
%%%%%%%%%%%%%%%%%%%%%%%%



\begin{document}

%%%%%%%%%%%%%%%%%%%%%%%%
%% Título e cabeçalho
%\noindent\parbox[c]{.15\textwidth}{\includegraphics[width=.15\textwidth]{logo}}\hfill
\parbox[c]{.825\textwidth}{\raggedright%
  \sffamily {\LARGE

Cálculo Numérico: Notas de aula: Bissecção

\par\bigskip}
{Prof: Felipe Figueiredo\par}
{\url{http://sites.google.com/site/proffelipefigueiredo}}

\vspace{1cm}
}
%%%%%%%%%%%%%%%%%%%%%%%%


%%%%%%%%%%%%%%%%%%%%%%%%
\section*{2.3.2 I Método da Bissecção (pg41)}

\begin{itemize}
\item Começar a aula com jogo de adivinhação
\item Motivar o método da bissecção para otimizar o processo
\end{itemize}

\subsection*{Jogo}

\begin{itemize}
\item Escolher dois voluntários na turma
\item O primeiro vai fazer tentativas de adivinhar o número escolhido
\item O segundo sabe qual é o número escolhido mas só pode dar pistas
  se a tentativa é maior ou menor que a resposta
\item Contar quantas tentativas são feitas.
\item Parâmetros iniciais: número 71, maior que 0, menor que 200
\end{itemize}

\subsection*{Bissecção}

Reduzir o intervalo inicial $[a,b]$ para intervalos $[a_k,b_k]$ até
chegar na precisão desejada ($b_k - a_b < \varepsilon$). Cada iteração
$k$ considera o ponto médio $x_k$ como valor aproximado


Exemplos: 

\begin{enumerate}
\item Função de primeiro grau $f(x) = -x+1$ no intervalo $[0,3]$
  $$[a,b]=[0,3]=[a_0,b_0]$$
  $$f(a_0)=1>0, f(b_0)=-2<0$$
  $$x_0 = 1.5$$
  $$f(x)$$


\item Encontrar raiz da função $f(x) = x^2 - 2x +1$ no intervalo $[0,5]$
  $$[a,b]=[0,5]=[a_0,b_0]$$
  $$f(a_0)=1>0, f(b_0)>0$$
  $$x_0 = 2.5$$
  $$f(x)$$

\item Raiz de $f(x) = e^x \sin{x} $

\end{enumerate}


\end{document}
