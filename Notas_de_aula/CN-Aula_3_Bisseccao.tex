\everymath{\displaystyle}
%\documentclass[pdftex,a4paper]{article}
\documentclass[a4paper]{article}
%%classes: article, report, book, proc, amsproc

%%%%%%%%%%%%%%%%%%%%%%%%
%% Misc
% para acertar os acentos
\usepackage[brazilian]{babel} 
%\usepackage[portuguese]{babel} 
% \usepackage[english]{babel}
% \usepackage[T1]{fontenc}
% \usepackage[latin1]{inputenc}
\usepackage[utf8]{inputenc}
\usepackage{indentfirst}
\usepackage{fullpage}
% \usepackage{graphicx} %See PDF section
\usepackage{multicol}
\setlength{\columnseprule}{0.5pt}
\setlength{\columnsep}{20pt}
%%%%%%%%%%%%%%%%%%%%%%%%
%%%%%%%%%%%%%%%%%%%%%%%%
%% PDF support

\usepackage[pdftex]{color,graphicx}
% %% Hyper-refs
\usepackage[pdftex]{hyperref} % for printing
% \usepackage[pdftex,bookmarks,colorlinks]{hyperref} % for screen

%% \newif\ifPDF
%% \ifx\pdfoutput\undefined\PDFfalse
%% \else\ifnum\pdfoutput > 0\PDFtrue
%%      \else\PDFfalse
%%      \fi
%% \fi

%% \ifPDF
%%   \usepackage[T1]{fontenc}
%%   \usepackage{aeguill}
%%   \usepackage[pdftex]{graphicx,color}
%%   \usepackage[pdftex]{hyperref}
%% \else
%%   \usepackage[T1]{fontenc}
%%   \usepackage[dvips]{graphicx}
%%   \usepackage[dvips]{hyperref}
%% \fi

%%%%%%%%%%%%%%%%%%%%%%%%


%%%%%%%%%%%%%%%%%%%%%%%%
%% Math
\usepackage{amsmath,amsfonts,amssymb}
% para usar R de Real do jeito que o povo gosta
\usepackage{amsfonts} % \mathbb
% para usar as letras frescas como L de Espaco das Transf Lineares
% \usepackage{mathrsfs} % \mathscr

% Oferecer seno e tangente em pt, com os comandos usuais.
\providecommand{\sin}{} \renewcommand{\sin}{\hspace{2pt}\mathrm{sen}}
\providecommand{\tan}{} \renewcommand{\tan}{\hspace{2pt}\mathrm{tg}}

% dt of integrals = \ud t
\newcommand{\ud}{\mathrm{\ d}}
%%%%%%%%%%%%%%%%%%%%%%%%



\begin{document}

%%%%%%%%%%%%%%%%%%%%%%%%
%% Título e cabeçalho
%\noindent\parbox[c]{.15\textwidth}{\includegraphics[width=.15\textwidth]{logo}}\hfill
\parbox[c]{.825\textwidth}{\raggedright%
  \sffamily {\LARGE

Cálculo Numérico: Notas de aula: Bissecção

\par\bigskip}
{Prof: Felipe Figueiredo\par}
{\url{http://sites.google.com/site/proffelipefigueiredo}}

\vspace{1cm}
}
%%%%%%%%%%%%%%%%%%%%%%%%


%%%%%%%%%%%%%%%%%%%%%%%%
\section*{2.3.2 I Método da Bissecção (pg41)}

\subsection*{Isolamento de raízes}
``Chutar'' valores para $x$ e verificar o sinal de $f(x)$, construindo
uma tabela.

Exemplos:

$$f(x)=-x+1$$

$$\begin{tabular}{c|ccccc}
%  \hline
  $x$ & -4 & -2 & 0 & 2 & 4\\
  \hline
  $f(x)$ & + & + & + & - & - \\
%  \hline
\end{tabular}$$

\bigskip
Já na função:

$$f(x) = x^2-1$$

$$\begin{tabular}{c|ccccc}
  $x$ & -4 & -2 & 0 & 2 & 4\\
  \hline
  $f(x)$ & + & + & - & + & +\\
\end{tabular}$$

% $$f(x) = x^3-9x +3$$

% $$\begin{tabular}{c|cccccccc}
% %  \hline
%   $x$ & -5 & -3 & -1 & 0 & 1 & 2 & 3 & 4 \\
%   \hline
%   $f(x)$&- & + & + & + & - & - & + & + \\
% %  \hline
% \end{tabular}$$

\subsection*{Jogo}

Atividade lúdica: Fazer um jogo de adivinhação e motivar o método da
bissecção para otimizar o processo.


\begin{itemize}
\item Escolher dois voluntários na turma.
\item O primeiro vai fazer tentativas de adivinhar o número escolhido
\item O segundo sabe qual é o número escolhido mas só pode dar pistas
  se a tentativa é maior ou menor que a resposta
\item O prof. vai contar quantas tentativas são feitas.
\item Exemplo: número 71, maior que 0, menor que 100
\end{itemize}

\clearpage
\subsection*{Método da Bissecção}

Reduzir o intervalo inicial $[a,b]$ para intervalos $[a_k,b_k]$ até
chegar na precisão desejada ($b_k - a_b < \varepsilon$). Cada iteração
$k$ considera o ponto médio $x_k$ como valor aproximado $\bar{x}$.

\bigskip
\begin{center}
  \fbox{
  % \addtolength{\linewidth}{-2\fboxsep}%
  % \addtolength{\linewidth}{-2\fboxrule}%
    \begin{minipage}{0.3\linewidth}
      {\bf Teste}

      Se $f(a)f(x)>0$, então $a=x$.  

      Caso contrário, então $b=x$.
    \end{minipage}
  }
\end{center}
\bigskip

Exemplos: 

\begin{enumerate}
\item Função de primeiro grau $f(x) = -x + 1$ no intervalo $[0,4]$

  $$\begin{tabular}{|c|c|c|c|c|c|}
    \hline
    $k$ & $x$ & $f(x)$ & $a$ & $b$ & $b-a$\\
    \hline
    0 & ? & ? & 0 & 4 & 4\\
    \hline
    1 &  2 & -1 & 0 & 2 & 2\\
    \hline
    2 &  1 & 0 &0 & 1 & 1 \\
    \hline
  \end{tabular}$$

  \begin{displaymath}    
    [a,b]=[0,4]=[a_0,b_0]
  \end{displaymath}
  \begin{displaymath}    
    f(a_0)=1>0, f(b_0)=-3<0
  \end{displaymath}  
  \begin{displaymath}    
    x_0 = 2
  \end{displaymath}
  \begin{displaymath}    
    f(0)f(2)<0 \text{ e }
    f(2)f(4)>0 \Rightarrow \text{ escolhemos } [0,2]
  \end{displaymath}
  \begin{displaymath}
    x_1 = 1
  \end{displaymath}
  \begin{displaymath}
    f(1) = 0
  \end{displaymath}
  %   \begin{align*}
  %     [a,b]=[0,4]=[a_0,b_0]\\
  %     f(a_0)=1>0, f(b_0)=-3<0\\
  %     x_0 = 2\\
  %     f(0)f(2)<0 \text{ e }\\
  %     f(2)f(4)>0 \Rightarrow \text{ escolhemos } [0,2]\\
  % \end{align*}

Fim! Encontramos a raíz exata! (Isso nunca vai mais acontecer...)

\item Encontrar raiz da função $f(x) = x^2 - 1$ no intervalo $[0,3]$
  com precisão de $10^{-1}$
  $$[a,b]=[0,3]=[a_0,b_0]$$
  $$f(0)=-1<0 \text{ e } f(3)=8>0$$
  $$x_0 = 1.5$$
  $$f(1.5) = 1.25$$

  $$\begin{tabular}{|c|c|c|c|c|c|}
    \hline
    $k$ & $\bar{x}$ & $f(\bar{x})$ & $a$ & $b$ & $b-a$\\
    \hline
    0 & ? & ? & 0 & 3 & 3\\
    \hline
    1 & 1.5 & 1.25 & 0 & 1.5 & 1.5\\
    \hline
    2 & 0.75 & -0.4375 & 0.75  & 1.5 & 0.75  \\
    \hline
    3 & 1.125 & 0.265625 & 0.75 & 1.125 & 0.375 \\
    \hline
    4 & 0.9375 & -0.12109375 & 0.9375 & 1.125 & 0.1875 \\
    \hline
    5 & 1.03125 & 0.063476563 & 0.9375 & 1.03125 & 0.09375 $<$ 0.1 \\
    \hline
  \end{tabular}$$

Então $\bar{x}= 1.0$ com $k=5$ iterações.

  % \begin{tabular}{|c|c|c|c|c|c|}
  %   \hline
  %   $k$ & $\bar{x}$ & $f(\bar{x})$ & $a$ & $b$ & $b-a$\\
  %   \hline
  %   0 & ? & ? & 0 & 3 & 3\\
  %   \hline
  %   1 & 1.5 & 1.25 & 0 & 1.5 & 1.5\\
  %   \hline
  %   2 & 0.8 & -0.36 & 0.8 & 1.5 & 0.7 \\
  %   \hline
  %   3 & 1.1 & 0.21 & 0.8 & 1.1 & 0.3 \\
  %   \hline
  %   4 & 1.0 & 0.0 & ? & ? &  \\
  %   \hline
  %   5 & 1.03125 & 0.063476563 & 0.9375 & 1.03125 & 0.09375 $<$ 0.1 \\
  %   \hline
  % \end{tabular}


% \item Raiz de $f(x) = e^x \sin{x} $

\end{enumerate}


\end{document}
