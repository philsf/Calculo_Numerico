\everymath{\displaystyle}
%\documentclass[pdftex,a4paper]{article}
\documentclass[a4paper]{article}
%%classes: article, report, book, proc, amsproc

%%%%%%%%%%%%%%%%%%%%%%%%
%% Misc
% para acertar os acentos
\usepackage[brazilian]{babel} 
%\usepackage[portuguese]{babel} 
% \usepackage[english]{babel}
% \usepackage[T1]{fontenc}
% \usepackage[latin1]{inputenc}
\usepackage[utf8]{inputenc}
\usepackage{indentfirst}
\usepackage{fullpage}
% \usepackage{graphicx} %See PDF section
\usepackage{multicol}
\setlength{\columnseprule}{0.5pt}
\setlength{\columnsep}{20pt}
%%%%%%%%%%%%%%%%%%%%%%%%
%%%%%%%%%%%%%%%%%%%%%%%%
%% PDF support

\usepackage[pdftex]{color,graphicx}
% %% Hyper-refs
\usepackage[pdftex]{hyperref} % for printing
% \usepackage[pdftex,bookmarks,colorlinks]{hyperref} % for screen

%% \newif\ifPDF
%% \ifx\pdfoutput\undefined\PDFfalse
%% \else\ifnum\pdfoutput > 0\PDFtrue
%%      \else\PDFfalse
%%      \fi
%% \fi

%% \ifPDF
%%   \usepackage[T1]{fontenc}
%%   \usepackage{aeguill}
%%   \usepackage[pdftex]{graphicx,color}
%%   \usepackage[pdftex]{hyperref}
%% \else
%%   \usepackage[T1]{fontenc}
%%   \usepackage[dvips]{graphicx}
%%   \usepackage[dvips]{hyperref}
%% \fi

%%%%%%%%%%%%%%%%%%%%%%%%


%%%%%%%%%%%%%%%%%%%%%%%%
%% Math
\usepackage{amsmath,amsfonts,amssymb}
% para usar R de Real do jeito que o povo gosta
\usepackage{amsfonts} % \mathbb
% para usar as letras frescas como L de Espaco das Transf Lineares
% \usepackage{mathrsfs} % \mathscr

% Oferecer seno e tangente em pt, com os comandos usuais.
\providecommand{\sin}{} \renewcommand{\sin}{\hspace{2pt}\mathrm{sen}}
\providecommand{\tan}{} \renewcommand{\tan}{\hspace{2pt}\mathrm{tg}}

% dt of integrals = \ud t
\newcommand{\ud}{\mathrm{\ d}}
%%%%%%%%%%%%%%%%%%%%%%%%

\begin{document}

%%%%%%%%%%%%%%%%%%%%%%%%
%% Título e cabeçalho
%\noindent\parbox[c]{.15\textwidth}{\includegraphics[width=.15\textwidth]{logo}}\hfill
\parbox[c]{.825\textwidth}{\raggedright%
  \sffamily {\LARGE

Cálculo Numérico: Notas de Aula: Método de Newton

\par\bigskip}
{Prof: Felipe Figueiredo\par}
{\url{http://sites.google.com/site/proffelipefigueiredo}\par}
}

Versão: \verb|20150527|

%%%%%%%%%%%%%%%%%%%%%%%%


%%%%%%%%%%%%%%%%%%%%%%%%
\section*{Revisão de Álgebra Linear}

\subsection*{Matrizes Triangulares}
\begin{displaymath}
  L = \begin{bmatrix}
    1&0&0&0\\
    0&1&0&0\\
    1&0&1&0\\
    0&1&0&1\\
  \end{bmatrix}
\end{displaymath}

e 

\begin{displaymath}
  U = \begin{bmatrix}
    -1&0&2&1\\
    0&1&1&0\\
    0&0&-1&3\\
    0&0&0&4\\
  \end{bmatrix}
\end{displaymath}

\subsection*{Produto de Matrizes Triangulares}

\begin{displaymath}
  LU = \begin{bmatrix}
    1&0&0&0\\
    0&1&0&0\\
    1&0&1&0\\
    0&1&0&1\\
  \end{bmatrix} \begin{bmatrix}
    -1&0&2&1\\
    0&1&1&0\\
    0&0&-1&3\\
    0&0&0&4\\
  \end{bmatrix}
  =
  \begin{bmatrix}
    -1&0&2&1\\
    0&1&1&0\\
    -1&0&1&4\\
    0&1&1&4\\
  \end{bmatrix}
\end{displaymath}

\subsection*{Sistemas Lineares e Matriciais}

\begin{displaymath}
  Ux=b
\end{displaymath}

\begin{displaymath}
  \begin{bmatrix}
    -1&0&2&1\\
    0&1&1&0\\
    0&0&-1&3\\
    0&0&0&4\\
  \end{bmatrix}
  \begin{bmatrix}
    x_1\\
    x_2\\
    x_3\\
    x_4\\
  \end{bmatrix}
  =
  \begin{bmatrix}
    -6\\
    1\\
    4\\
    4\\
  \end{bmatrix}
\end{displaymath}

\begin{displaymath}
  \left\{\begin{array}{ccccccccc}
      -x_1 &+&  &+& 2x_3 &+& x_4 &=&-6 \\
      && x_2 &+& x_3 && &=& 1\\
      && &-&x_3 &+& 3x_4 &=&4\\
      &&&&&&4x_4 &=& 4\\
    \end{array}
  \right.
\end{displaymath}

\subsection*{Resolução de Sistemas Triangulares}

\begin{displaymath}
  x=
  \begin{bmatrix}
    x_1\\
    x_2\\
    x_3\\
    x_4\\
  \end{bmatrix}
  =
\begin{bmatrix}
    5\\
    2\\
    -1\\
    1\\
  \end{bmatrix}
\end{displaymath}


\subsection*{Inversão de Matrizes}

Resolução de um sistema linear usando a matriz inversa

\begin{displaymath}
  Ax=b
\end{displaymath}
\begin{displaymath}
  A^{-1}Ax=A^{-1}b
\end{displaymath}
\begin{displaymath}
  x=A^{-1}b
\end{displaymath}


\begin{displaymath}
  A=\begin{bmatrix}
  2&3\\
  3&4\\
  \end{bmatrix}
\end{displaymath}

\begin{displaymath}
  [A|I]=\begin{bmatrix}
  2&3&|&1&0\\
  3&4&|&0&1\\
  \end{bmatrix}
\end{displaymath}

\begin{displaymath}
  \begin{bmatrix}
    1&3/2&|&1/2&0\\
    3&4&|&0&1\\
  \end{bmatrix}
\end{displaymath}

\begin{displaymath}
  \begin{bmatrix}
    1&3/2&|&1/2&0\\
    0&1/2&|&3/2&-1\\
  \end{bmatrix}
\end{displaymath}

\begin{displaymath}
  \begin{bmatrix}
    1&3/2&|&1/2&0\\
    0&1&|&3&-2\\
  \end{bmatrix}
\end{displaymath}

\begin{displaymath}
  \begin{bmatrix}
    1&0&|&-4&3\\
    0&1&|&3&-2\\
  \end{bmatrix}
\end{displaymath}


\begin{displaymath}
  A^{-1}=
  \begin{bmatrix}
    -4&3\\
    3&-2\\
  \end{bmatrix}
\end{displaymath}
\end{document}
