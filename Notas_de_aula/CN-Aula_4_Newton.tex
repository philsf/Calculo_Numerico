\everymath{\displaystyle}
%\documentclass[pdftex,a4paper]{article}
\documentclass[a4paper]{article}
%%classes: article, report, book, proc, amsproc

%%%%%%%%%%%%%%%%%%%%%%%%
%% Misc
% para acertar os acentos
\usepackage[brazilian]{babel} 
%\usepackage[portuguese]{babel} 
% \usepackage[english]{babel}
% \usepackage[T1]{fontenc}
% \usepackage[latin1]{inputenc}
\usepackage[utf8]{inputenc}
\usepackage{indentfirst}
\usepackage{fullpage}
% \usepackage{graphicx} %See PDF section
\usepackage{multicol}
\setlength{\columnseprule}{0.5pt}
\setlength{\columnsep}{20pt}
%%%%%%%%%%%%%%%%%%%%%%%%
%%%%%%%%%%%%%%%%%%%%%%%%
%% PDF support

\usepackage[pdftex]{color,graphicx}
% %% Hyper-refs
\usepackage[pdftex]{hyperref} % for printing
% \usepackage[pdftex,bookmarks,colorlinks]{hyperref} % for screen

%% \newif\ifPDF
%% \ifx\pdfoutput\undefined\PDFfalse
%% \else\ifnum\pdfoutput > 0\PDFtrue
%%      \else\PDFfalse
%%      \fi
%% \fi

%% \ifPDF
%%   \usepackage[T1]{fontenc}
%%   \usepackage{aeguill}
%%   \usepackage[pdftex]{graphicx,color}
%%   \usepackage[pdftex]{hyperref}
%% \else
%%   \usepackage[T1]{fontenc}
%%   \usepackage[dvips]{graphicx}
%%   \usepackage[dvips]{hyperref}
%% \fi

%%%%%%%%%%%%%%%%%%%%%%%%


%%%%%%%%%%%%%%%%%%%%%%%%
%% Math
\usepackage{amsmath,amsfonts,amssymb}
% para usar R de Real do jeito que o povo gosta
\usepackage{amsfonts} % \mathbb
% para usar as letras frescas como L de Espaco das Transf Lineares
% \usepackage{mathrsfs} % \mathscr

% Oferecer seno e tangente em pt, com os comandos usuais.
\providecommand{\sin}{} \renewcommand{\sin}{\hspace{2pt}\mathrm{sen}}
\providecommand{\tan}{} \renewcommand{\tan}{\hspace{2pt}\mathrm{tg}}

% dt of integrals = \ud t
\newcommand{\ud}{\mathrm{\ d}}
%%%%%%%%%%%%%%%%%%%%%%%%



\begin{document}

%%%%%%%%%%%%%%%%%%%%%%%%
%% Título e cabeçalho
%\noindent\parbox[c]{.15\textwidth}{\includegraphics[width=.15\textwidth]{logo}}\hfill
\parbox[c]{.825\textwidth}{\raggedright%
  \sffamily {\LARGE

Cálculo Numérico: Notas de Aula: Método de Newton

\par\bigskip}
{Prof: Felipe Figueiredo\par}
{\url{http://sites.google.com/site/proffelipefigueiredo}\par}
}

Versão: \verb|20150328|

%%%%%%%%%%%%%%%%%%%%%%%%


%%%%%%%%%%%%%%%%%%%%%%%%
\section*{2.3.2 IV Método de Newton-Raphson (pg 66)}

$$
x_{k+1} = x_k - \frac{f(x_k)}{f'(x_k)}
$$

Exemplos:

\begin{displaymath}
  f(x) = -x+1, x_0=2
\end{displaymath}


Preparando o terreno, temos:

\begin{displaymath}
  f'(x) = -1
\end{displaymath}
portanto
\begin{displaymath}
  f(x_0) = -2+1 = -1, f'(x_0) = -1
\end{displaymath}

Aplicando a fórmula, temos:

\begin{displaymath}
  \bar{x}_1 = \bar{x}_0 - \frac{f(\bar{x}_0)}{f'(\bar{x}_0)} = 2-\frac{-1}{-1} = 1
\end{displaymath}


\begin{center}
  \begin{tabular}{c|c|c|c|c}
    k & $\bar{x}$ & $f(\bar{x})$ & $f'(\bar{x})$ & $|\bar{x}_k - \bar{x}_{k-1}|$\\
    \hline
    0 & 2 & -1 & -1 & ? \\
    \hline
    1 & 1 & 0 & ? & 0 \\
  \end{tabular}
\end{center}

FIM!

Agora, com outro exemplo:

\begin{displaymath}
  f(x) = x^2 -1, x_0 = 1.5, \varepsilon <10^{-1}
\end{displaymath}

Para simplificar as contas, como a precisão exigida é de apenas um
dígito ($\varepsilon <10^{-1})$, podemos usar dois dígitos nas etapas.

Preparando o terreno:

\begin{displaymath}
  f'(x) = 2x
\end{displaymath}

portanto,

\begin{displaymath}
  f(\bar{x}_0) = 1.5^2-1 = 1.25, f'(\bar{x}_0) = 2 \times 1.5 = 3
\end{displaymath}

Aplicando a fórmula, temos:

\begin{displaymath}
  \bar{x}_1 = 1.5 - \frac{1.25}{3} = 1.08333 \approx 1.08
\end{displaymath}

\begin{center}
  \begin{tabular}{c|c|c|c|c}
    k & $\bar{x}$ & $f(\bar{x})$ & $f'(\bar{x})$ & $|\bar{x}_k - \bar{x}_{k-1}|$\\
    \hline
    0 & 1.50 & 1.25 & 3.00 & ? \\
    \hline
    1 & 1.08 & 0.17 & 2.16 & 0.42 \\
    \hline
    2 & 1.00 &  &  & $0.08 < 0.1$ \\
  \end{tabular}
\end{center}

FIM!

\end{document}
